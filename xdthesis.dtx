%% \iffalse
%%  Local Variables:
%%  mode: doctex
%%  TeX-master: t
%%  End:
%% \fi
%% \iffalse meta-comment
%%
%% Copyright (C) 2008-2011 by gprsnl <gprsnl@yeah.net>
%%
%% This file may be distributed and/or modified under the
%% conditions of the LaTeX Project Public License, either version 1.3a
%% of this license or (at your option) any later version.
%% The latest version of this license is in:
%%
%% http://www.latex-project.org/lppl.txt
%%
%% and version 1.3a or later is part of all distributions of LaTeX
%% version 2004/10/01 or later.
%%
%% \fi
%
% \CheckSum{1025}
%
% \iffalse
%<*driver>
\ProvidesFile{xdthesis.dtx}[2012/02/07 0.9.4a Xidian University Thesis Template]
\documentclass[10pt]{ltxdoc}
\usepackage{url}
\usepackage{xltxtra}
\usepackage{xeCJK}
\usepackage[normalem]{ulem}
\setCJKmainfont[BoldFont={Adobe Heiti Std}]{Adobe Song Std}
\usepackage{indentfirst}
\setlength{\parskip}{4pt plus1pt minus0pt}
\setlength{\topsep}{0pt}
\setlength{\partopsep}{0pt}
\setlength{\parindent}{20pt}
\addtolength{\oddsidemargin}{-1cm}
\advance\textwidth 1.5cm
\addtolength{\topmargin}{-1cm}
\addtolength{\headsep}{0.3cm}
\addtolength{\textheight}{2.3cm}
\renewcommand{\baselinestretch}{1.3}
% \DefineVerbatimEnvironment{example}{Verbatim}%
%   {frame=single, framerule=0.3mm, rulecolor=\color{red!75!green!50!blue}, 
%    fillcolor=\color{red!75!green!50!blue!15},framesep=2mm, baselinestretch=1.2,
%    fontsize=\small, gobble=1}
% \DefineVerbatimEnvironment{shell}{Verbatim}%
%   {frame=single, framerule=0.3mm, rulecolor=\color{red!85!green!60}, 
%    fillcolor=\color{red!85!green!10},framesep=2mm,fontsize=\small, gobble=1}
\makeatletter
\def\DescribeOption#1{\SpecialOptionIndex{#1}}
\def\tableofcontents{\renewcommand{\baselinestretch}{1.0}\@starttoc{toc}}
\def\DescribeMacro{\Describe@Macro}
\def\Describe@Macro#1{\PrintDescribeMacro{#1}\SpecialUsageIndex{#1}}
\def\PrintDescribeMacro#1{{\color{-red!75!green!50!blue!55}\MacroFont \string #1\hskip1em }}
\def\ps@headings{%
  \let\@oddfoot\@empty
  \def\@oddhead{\vbox{\hbox
    to\textwidth{\llap{\fbox{\rightmark\rule[-2pt]{0pt}{13pt}}}\hfil\thepage}\vskip-0.7pt
      \hbox to \textwidth{\hrulefill}}}%
  \let\@evenfoot\@oddfoot
  \let\@evenhead\@oddhead
  \let\@mkboth\markboth
  \def\sectionmark##1{%
    \markright{\ifnum \c@secnumdepth >\m@ne
      \thesection\quad
      \fi
      ##1}}
  \def\subsectionmark##1{%
    \markright{\ifnum \c@secnumdepth >\m@ne
      \thesubsection\quad
      \fi
      ##1}}
  \def\subsubsectionmark##1{%
    \markright{\ifnum \c@secnumdepth >\m@ne
      \thesubsubsection\quad
      \fi
      ##1}}}
\renewcommand\section{\@startsection {section}{1}{\z@}%
                                   {-3.5ex \@plus -1ex \@minus -.2ex}%
                                   {2.3ex \@plus.2ex}%
                                   {\normalfont\Large\bfseries}}
\renewcommand\subsection{\@startsection{subsection}{2}{\z@}%
                                     {-3.25ex\@plus -1ex \@minus -.2ex}%
                                     {1.5ex \@plus .2ex}%
                                     {\normalfont\large\bfseries}}
\renewcommand\subsubsection{\@startsection{subsubsection}{3}{\z@}%
                                     {-3.25ex\@plus -1ex \@minus -.2ex}%
                                     {1.5ex \@plus .2ex}%
                                     {\normalfont\normalsize\bfseries}}
\renewcommand\paragraph{\@startsection{paragraph}{4}{\z@}%
                                    {3.25ex \@plus1ex \@minus.2ex}%
                                    {-1em}%
                                    {\normalfont\normalsize\bfseries}}
\renewcommand\subparagraph{\@startsection{subparagraph}{5}{\parindent}%
                                       {3.25ex \@plus1ex \@minus .2ex}%
                                       {-1em}%
                                      {\normalfont\normalsize\bfseries}}
\makeatother
\pagestyle{empty}
\EnableCrossrefs
\CodelineIndex
\RecordChanges
\begin{document}
\DocInput{xdthesis.dtx}
\clearpage
\end{document}
%</driver>
% \fi
%
% \GetFileInfo{xdthesis.dtx}
% \MakeShortVerb{\|}
%
% \changes{v0.9.4}{2012/01/10}{Made the full support for the master option.}
% \changes{v0.9.3a}{2011/12/13}{Fixed line space and chapter space in TOC.}
% \changes{v0.9.3}{2011/04/20}{Fixed the error on \texttt{$\backslash$nobreakspace}.}
% \changes{v0.9.2}{2009/06/17}{Using full Chinese punctuation.}
% \changes{v0.9.1}{2009/06/15}{Template for Bachelor's thesis release candidate.}
% \changes{v0.9}{2009/06/10}{Template realy works for Bachelor's thesis.}
% \changes{v0.2}{2009/06/06}{Defined styles of fonts, names, and titles and floats formats.}
% \changes{v0.1}{2008/06/24}{The \XeLaTeX\ template for writing thesis of Xidian starts.}
%
% \def\xdthesis{\textsc{XD}\-\textsc{Thesis}}
% \def\pkg#1{\texttt{#1}}
% \def\xdu{西安电子科技大学}
%
% \DoNotIndex{\begin,\end,\begingroup,\endgroup}
% \DoNotIndex{\ifx,\ifdim,\ifnum,\ifcase,\else,\or,\fi}
% \DoNotIndex{\let,\def,\xdef,\newcommand,\renewcommand}
% \DoNotIndex{\expandafter,\csname,\endcsname,\relax,\protect}
% \DoNotIndex{\Huge,\huge,\LARGE,\Large,\large,\normalsize}
% \DoNotIndex{\small,\footnotesize,\scriptsize,\tiny}
% \DoNotIndex{\normalfont,\bfseries,\slshape,\interlinepenalty}
% \DoNotIndex{\hfil,\par,\hskip,\vskip,\vspace,\quad}
% \DoNotIndex{\centering,\raggedright}
% \DoNotIndex{\c@secnumdepth,\@startsection,\@setfontsize}
% \DoNotIndex{\ ,\@plus,\@minus,\p@,\z@,\@m,\@M,\@ne,\m@ne}
% \DoNotIndex{\@@par,\DeclareOperation,\RequirePackage,\LoadClass}
% \DoNotIndex{\AtBeginDocument,\AtEndDocument}
%
% \IndexPrologue{\section*{索引}%
%    \addcontentsline{toc}{section}{索~~~~引}}
% \GlossaryPrologue{\section*{修改记录}%
%    \addcontentsline{toc}{section}{修改记录}}
%
% \renewcommand{\abstractname}{摘~~要}
% \renewcommand{\contentsname}{目~~录}
%
% \title{\xdthesis:\xdu{}学位论文模板\thanks{Xidian University \XeLaTeX{}
%     Thesis Template.}}
% \author{gprsnl\\[5pt]
%   \texttt{gprsnl@yeah.net}} \date{v\fileversion\ (\filedate)}
% \maketitle\thispagestyle{empty}
%
% \begin{abstract}\noindent
%   此宏包旨在建立一个简单易用的西安电子科技大学学位论文模板,包括本科毕业设计、
%   硕士及博士学位论文。目前正在开发本科毕业设计的模板,对其它格式的支持会陆续加
%   入。
% \end{abstract}
%
% \vskip2cm 
% \def\abstractname{免责声明}
% \begin{abstract}
% \noindent
% \begin{enumerate}
% \item 本模板的发布遵守~\LaTeX{} Project Public License,使用前请认真阅读协议内容。
% \item 本模板为作者根据\xdu{}教务处颁发的《本科生毕业设计(论文)工作手册》编写
%   而成,旨在供\xdu{}毕业生撰写学位论文使用。
% \item 此模板仅为写作指南的参考实现,不保证格式审查老师不提意见。任何由于使用本
%   模板而引起的论文格式审查问题均与本模板作者无关。
% \item 任何个人或组织以本模板为基础进行修改、扩展而生成的新的专用模板,请严格遵
%   守~\LaTeX{} Project Public License 协议。由于违犯协议而引起的任何纠纷争端均与
%   本模板作者无关。
% \end{enumerate}
% \end{abstract}
%
% \clearpage
% \begin{multicols}{2}[
% \section*{\contentsname}
% \setlength{\columnseprule}{.4pt}
% \setlength{\columnsep}{18pt}]
%  \tableofcontents
% \end{multicols}
%
% \clearpage
% \pagenumbering{arabic}
% \pagestyle{headings}
% \section{模板介绍}
%
% \section{安装}
% \label{sec:installation}
%
% \subsection{下载}
% \xdthesis{} 的主页是:  \url{http://code.google.com/p/xdthesis/}。
%
%\section{致谢}
%\label{sec:thanks}
%
% \StopEventually{\PrintChanges\PrintIndex}
% \clearpage
%
% \section{实现细节}
%
% \subsection{基本信息}
%    \begin{macrocode}
%<cls>\NeedsTeXFormat{LaTeX2e}[1999/12/01]
%<cls>\ProvidesClass{xdthesis}
%<cfg>\ProvidesFile{xdthesis.cfg}
%<cls|cfg>[2009/06/06 0.2 Xidian University Thesis Template]
%    \end{macrocode}
%
% \subsection{定义选项}
% \label{sec:defoption}
%
% \changes{v0.9.4a}{2012/02/07}{Added option print.}
% 定义论文类型以及是否涉密
%    \begin{macrocode}
%<*cls>
\hyphenation{XD-Thesis}
\def\xdthesis{\textsc{XDThesis}}
\def\version{0.9.4}
\newif\ifxd@bachelor\xd@bachelorfalse
\newif\ifxd@master\xd@masterfalse
\newif\ifxd@doctor\xd@doctorfalse
\newif\ifxd@secret\xd@secretfalse
\newif\ifxd@mkabstract\xd@mkabstractfalse
\newif\ifxd@numbered\xd@numberedtrue
\newif\ifxd@print\xd@printfalse
\DeclareOption{bachelor}{\xd@bachelortrue}
\DeclareOption{master}{\xd@mastertrue}
\DeclareOption{doctor}{\xd@doctortrue}
\DeclareOption{secret}{\xd@secrettrue}
\DeclareOption{print}{\xd@printtrue}
\AtEndOfClass{%
  \ifxd@bachelor\relax\else
    \ifxd@master\relax\else
      \ifxd@doctor\relax\else
        \ClassError{xdthesis}%
                   {Please specify a thesis option: bachelor, master or doctor.}{}
      \fi
    \fi
  \fi}
%    \end{macrocode}
%
%    \begin{macrocode}
\ExecuteOptions{arialtitle}
\ProcessOptions
\LoadClass[12pt, a4paper, openright]{book}
%</cls>
%    \end{macrocode}
%
% \subsection{装载宏包}
% \label{sec:loadpackage}
%
% \changes{v0.9.4a}{2012/02/07}{Added package calc for paper layout setup.}
% 设置页边距等引用宏包
%    \begin{macrocode}
%<*cls>
\RequirePackage{calc}
%    \end{macrocode}
% 参考文献引用宏包。
%    \begin{macrocode}
\RequirePackage[numbers,super,sort&compress]{natbib}
%    \end{macrocode}
%
%    \begin{macrocode}
\ifxd@print
\RequirePackage[colorlinks=true,allcolors=black]{hyperref}
\else
\RequirePackage[colorlinks=true]{hyperref}
\fi
%    \end{macrocode}
% 引用的宏包和相应的定义。
% \pkg{hypernat} 让~\pkg{hyperref} 和~\pkg{natbib} 协调的工作。应该
% 在~\pkg{natbib} 和~\pkg{hyperref} 之后加载,参看其文档。
%    \begin{macrocode}
% \RequirePackage{hypernat}
%    \end{macrocode}
%
% 首行缩进。
%    \begin{macrocode}
\RequirePackage{indentfirst}
%    \end{macrocode}
%
% \changes{v0.9.1}{2009/06/15}{引用\pkg{paralist},缩小列表环境的行距。}
% 更好的列表环境。
%    \begin{macrocode}
\RequirePackage[neverdecrease]{paralist}
%    \end{macrocode}
%
% 页眉页脚。
%    \begin{macrocode}
% \RequirePackage{fancyhdr}
%    \end{macrocode}
%
% AMS\LaTeX{} 宏包,用来排出更加漂亮的公式
%    \begin{macrocode}
\RequirePackage{amsmath, amssymb}
%    \end{macrocode}
%
% 图形支持宏包。
%    \begin{macrocode}
\RequirePackage{graphicx}
%    \end{macrocode}
%
% 并排图形。\pkg{subfigure} 已经不再推荐,用新的~\pkg{subfig}。加入~|config| 选项以便兼容
% ~\pkg{subfigure} 的命令。
% 浮动图形和表格标题样式。\pkg{caption2} 已经不推荐使用,采用新的~\pkg{caption}。它会自动被
% ~\pkg{subfig} 装载进来。所以可以在后面看到~\cs{captionsetup} 的命令。
%    \begin{macrocode}
\RequirePackage{subfig}
% 下划线(虚线)
% \dashuline{dashing} dashed underline like dashing
\RequirePackage[normalem]{ulem}
%    \end{macrocode}
%
% 载入标题格式宏包。
%    \begin{macrocode}
\RequirePackage{ifthen}
\RequirePackage{titlesec,titletoc}
%    \end{macrocode}
% 本模板是基于\XeLaTeX{}的。
%    \begin{macrocode}
\RequirePackage[CJKnumber,BoldFont]{xeCJK}
% \changes{v0.9.1}{2009/06/17}{使用全角格式中文标点符号。}
\punctstyle{quanjiao}
\def\CJK@null{\kern\CJKnullspace\Unicode{48}{7}\kern\CJKnullspace}
\defaultfontfeatures{Mapping=tex-text} % after fontspec
\setCJKmainfont{Adobe Song Std}
\setCJKsansfont{Adobe Heiti Std}
% \setCJKmonofont{Adobe Kaiti Std}
\setCJKfamilyfont{song}{Adobe Song Std}
\setCJKfamilyfont{hei}{Adobe Heiti Std}
% \setCJKfamilyfont{fs}{Adobe Fangsong Std}
% \setCJKfamilyfont{kai}{Adobe Kaiti Std}
% \setCJKfamilyfont{li}{Adobe Kaiti Std} % todo: 用隶书字体代替
% \setCJKfamilyfont{you}{Adobe Kaiti Std} % todo: 用幼圆字体代替
\setmainfont{Times New Roman}
% \setsansfont{Arial}
\setmonofont{Courier Std}
% \changes{v0.9.3}{2011/04/20}{引用宏包顺序调整,以避免\texttt{nobreakspace}的问题。}
\RequirePackage{xunicode,xltxtra}
%</cls>
%    \end{macrocode}
%
% \subsection{主文档格式}
% \label{sec:mainbody}
% \subsubsection{Three matters}
%
%    \begin{macrocode}
%<*cls>
\renewcommand\frontmatter{%
  \if@openright\cleardoublepage\else\clearpage\fi
  \@mainmatterfalse
  \pagenumbering{roman}
  \pagestyle{xd@front}}
\renewcommand\mainmatter{%
  \if@openright\cleardoublepage\else\clearpage\fi
  \cleardoublepage
  \@mainmattertrue
  \pagenumbering{arabic}
  \pagestyle{xd@headings}}
\renewcommand\backmatter{%
  \if@openright\cleardoublepage\else\clearpage\fi
  \@mainmattertrue}
%</cls>
%    \end{macrocode}
%
% \subsubsection{字体}
% \label{sec:fonts}
% Ref 2:
% WORD 中的字号对应该关系如下:
% \begin{verbatim}
% 初号 = 42bp = 14.82mm = 42.1575pt
% 小初 = 36bp = 12.70mm = 36.135 pt
% 一号 = 26bp = 9.17mm = 26.0975pt
% 小一 = 24bp = 8.47mm = 24.09pt
% 二号 = 22bp = 7.76mm = 22.0825pt
% 小二 = 18bp = 6.35mm = 18.0675pt
% 三号 = 16bp = 5.64mm = 16.06pt
% 小三 = 15bp = 5.29mm = 15.05625pt
% 四号 = 14bp = 4.94mm = 14.0525pt
% 小四 = 12bp = 4.23mm = 12.045pt
% 五号 = 10.5bp = 3.70mm = 10.59375pt
% 小五 = 9bp = 3.18mm = 9.03375pt
% 六号 = 7.5bp = 2.56mm
% 小六 = 6.5bp = 2.29mm
% 七号 = 5.5bp = 1.94mm
% 八号 = 5bp = 1.76mm
%
% 1bp = 72.27/72 pt
% \end{verbatim}
%
%    \begin{macrocode}
%<*cls>
\newcommand{\song}{\CJKfamily{song}} % 宋体
\def\songti{\song}
\newcommand{\hei}{\CJKfamily{hei}} % 黑体
\def\heiti{\hei}
%    \end{macrocode}
%    \begin{macrocode}
\newlength\xd@linespace
\newcommand{\xd@choosefont}[2]{%
   \setlength{\xd@linespace}{#2*\real{#1}}%
   \fontsize{#2}{\xd@linespace}\selectfont}
\def\xd@define@fontsize#1#2{%
  \expandafter\newcommand\csname #1\endcsname[1][\baselinestretch]{%
    \xd@choosefont{##1}{#2}}}
\xd@define@fontsize{chuhao}{42bp}
\xd@define@fontsize{xiaochu}{36bp}
\xd@define@fontsize{yihao}{26bp}
\xd@define@fontsize{xiaoyi}{24bp}
\xd@define@fontsize{erhao}{22bp}
\xd@define@fontsize{xiaoer}{18bp}
\xd@define@fontsize{sanhao}{16bp}
\xd@define@fontsize{xiaosan}{15bp}
\xd@define@fontsize{sihao}{14bp}
\xd@define@fontsize{banxiaosi}{13bp}
\xd@define@fontsize{xiaosi}{12bp}
\xd@define@fontsize{dawu}{11bp}
\xd@define@fontsize{wuhao}{10.5bp}
\xd@define@fontsize{xiaowu}{9bp}
\xd@define@fontsize{liuhao}{7.5bp}
\xd@define@fontsize{xiaoliu}{6.5bp}
\xd@define@fontsize{qihao}{5.5bp}
\xd@define@fontsize{bahao}{5bp}
%    \end{macrocode}
%
% 定义行距,正文小四号(12pt)字,行距为1.5倍行距
%    \begin{macrocode}
\renewcommand\normalsize{\@setfontsize\normalsize{12bp}{18bp}}
\ifxd@bachelor
\renewcommand\baselinestretch{1.2}
\fi
\ifxd@master
\renewcommand\baselinestretch{1.2}
\fi
%</cls>
%    \end{macrocode}
%
% \subsubsection{页面设置}
% \label{sec:layout}
%
%    \begin{macrocode}
%<*cls>
\setlength{\textwidth}{\paperwidth}
\setlength{\textheight}{\paperheight}
\setlength\marginparwidth{0cm} 
\setlength\marginparsep{0cm}
\addtolength{\textwidth}{-6cm}
\setlength{\oddsidemargin}{4cm-1in}
\setlength{\evensidemargin}{2cm-1in}
\setlength{\topmargin}{1.45cm-1in} 
\setlength{\headheight}{20pt}
\setlength{\headsep}{0.6cm}
\setlength{\topskip}{0pt}
\setlength{\skip\footins}{15pt} 
\setlength{\footskip}{1.3cm}
\addtolength{\textheight}{-4.5cm}  
%</cls>
%    \end{macrocode}
%
% \subsubsection{页眉页脚}
% \label{sec:headerfooter}
%
% 新的一章最好从奇数页开始(openright),所以必须保证它前面那页如果没有内容也必
% 须没有页眉页脚。code stolen from fancyhdr
%    \begin{macrocode}
%<*cls>
\let\xd@cleardoublepage\cleardoublepage
\newcommand{\xd@clearemptydoublepage}{%
  \clearpage{\pagestyle{empty}\xd@cleardoublepage}}
\let\cleardoublepage\xd@clearemptydoublepage
\let\xd@orgtitle\title
\renewcommand{\title}[1]{\gdef\xd@title{#1}\xd@orgtitle{#1}}
%    \end{macrocode}
%
%
% 定义页眉和页脚。chapter 自动调用~thispagestyle{xd@front},所以要重新定
% 义~xd@front。
% \begin{macro}{\ps@xd@empty}
% \begin{macro}{\ps@xd@front}
% \begin{macro}{\ps@xd@headings}
% 定义页眉页脚格式:
% \begin{itemize}
% \item \texttt{xd@empty} :无页眉页脚
% \item \texttt{xd@front} :页眉中无页码
% \item \texttt{xd@headings}:页眉中显示页码
% \end{itemize}
%    \begin{macrocode}
\def\ps@xd@empty{%
  \let\@oddhead\@empty%
  \let\@evenhead\@empty%
  \let\@oddfoot\@empty%
  \let\@evenfoot\@empty}
\def\ps@xd@front{%
  \def\@oddhead{\vbox{\hbox to\textwidth{%
                \hfil{\wuhao\leftmark}\hfil}%
                \vskip2pt\rule{\textwidth}{0.75pt}}}%
  \def\@evenhead{\vbox{\hbox to\textwidth{%
                \hfil{\wuhao\rightmark}\hfil}%
                \vskip2pt\rule{\textwidth}{0.75pt}}}%
  \let\@oddfoot\@empty%
  \let\@evenfoot\@empty}
\def\ps@xd@headings{% 
  \def\@oddhead{\vbox{\hbox to\textwidth{%
                \hfil{\wuhao\leftmark}\hfil{\xiaowu\thepage}}%
                \vskip2pt\rule{\textwidth}{0.75pt}}}%
  \def\@evenhead{\vbox{\hbox to\textwidth{%
                {\xiaowu\thepage}\hfil{\wuhao\rightmark}\hfil}%
                \vskip2pt\rule{\textwidth}{0.75pt}}}%
  \let\@oddfoot\@empty%
  \let\@evenfoot\@empty}
%    \end{macrocode}
% \end{macro}
% \end{macro}
% \end{macro}
%
% 其实可以直接写到~\cs{chapter} 的定义里面。
%    \begin{macrocode} 
\renewcommand{\chaptermark}[1]{\markboth{\chaptername\hspace{2ex}#1}{\xd@title}}
\renewcommand{\sectionmark}[1]{\markright{\xd@title}}
%</cls>
%    \end{macrocode}
%
% \changes{v0.9.1}{2009/06/15}{增加首行按照两个中文字符缩进。}
% \subsubsection{段落}
% \label{sec:paragraph}
%
% 用于中文段落缩进和正文版式
%    \begin{macrocode}
%<*cls>
\newlength\CJK@twochars
\def\CJK@spaceChar{\Unicode{48}{7}}
\def\CJKindent{%
  \settowidth\CJK@twochars{\CJK@spaceChar\CJK@spaceChar}%
  \parindent\CJK@twochars}
%    \end{macrocode}
%
% 段落之间的竖直距离
%    \begin{macrocode}
\setlength{\parskip}{0pt \@plus2pt \@minus0pt}
%    \end{macrocode}
%
% 调整默认列表环境间的距离,以符合中文习惯。
% \begin{macro}{xd@item@space}
%    \begin{macrocode}
\def\xd@item@space{%
  \let\itemize\compactitem
  \let\enditemize\endcompactitem
  \let\enumerate\compactenum
  \let\endenumerate\endcompactenum
  \let\description\compactdesc
  \let\enddescription\endcompactdesc}
  
  
%
% \newcommand{\xddashuline}[1]{%
% \dashuline{\hbox to 4.45cm{\hfill{#1}\hfill}}
% }
% \def\XDBA@dashuline[#1]#2{%
  % \dashuline{\hbox to #1{\hfill#2\hfill}}}
% \def\XDBAdashuline{\@ifnextchar[\XDBA@dashuline\dashuline}
\def\xddashuline[#1]#2{%
  \dashuline{\hbox to #1{\hfill#2\hfill}}
}
\def\xduline[#1]#2{%
  \uline{\hbox to #1{\hfill#2\hfill}}
}
\newcommand\boldhei{%
\fontspec[BoldFont={Adobe Heiti Std},BoldFeatures={RawFeature={embolden=2}}]{Adobe Heiti Std}
}
\newcommand\boldsong{%
\fontspec[BoldFont={Adobe Song Std},BoldFeatures={RawFeature={embolden=2}}]{Adobe Song Std}
}

\def\xd@daihao{10701}
\def\xd@xuehao{0911120728}
\def\xd@leihao{TP391.41}
\def\xd@miji{公开}

\def\xd@timuI{论文题目}
\def\xd@timuII{}
\def\xd@timuIII{}
\def\xd@timuIIII{}
\def\xd@xingming{作者姓名}
\def\xd@jiaoshi{教师姓名职称}
\def\xd@menglei{学科门类}
\def\xd@zhuanye{学科专业}
\def\xd@riqi{日期}
	
\newcommand\daihao[1]{\def\xd@daihao{#1}}
\newcommand\xuehao[1]{\def\xd@xuehao{#1}}
\newcommand\leihao[1]{\def\xd@leihao{#1}}
\newcommand\miji[1]{\def\xd@miji{#1}}

\newcommand\timuI[1]{\def\xd@timuI{#1}}
\newcommand\timuII[1]{\def\xd@timuII{#1}}
\newcommand\timuIII[1]{\def\xd@timuIII{#1}}
\newcommand\timuIIII[1]{\def\xd@timuIIII{#1}}
\newcommand\xingming[1]{\def\xd@xingming{#1}}
\newcommand\jiaoshi[1]{\def\xd@jiaoshi{#1}}
\newcommand\menglei[1]{\def\xd@menglei{#1}}
\newcommand\zhuanye[1]{\def\xd@zhuanye{#1}}
\newcommand\riqi[1]{\def\xd@riqi{#1}}
	
\newcommand{\xdcover}{%

\ifxd@master
{
\cleardoublepage
\thispagestyle{empty}
{
%\setlength{\baselineskip}{30pt}
\bfseries\heiti\dawu
\makebox[1cm]{}\vspace{4mm}

\hspace{2mm}{代~~~~~~号}\xddashuline[4.3cm]{\boldhei\xd@daihao} \hspace*{1.4cm}
{学~~~~~~号}\xddashuline[4.3cm]{\boldhei\xd@xuehao} \vspace{3mm}

\hspace{2mm}{分~类~号}\xddashuline[4.3cm]{\boldhei\xd@leihao} \hspace*{1.4cm}
{密~~~~~~级}\xddashuline[4.3cm]{\xd@miji} \vspace{13.8cm}

\bfseries\sihao
\noindent\makebox[3.2cm][s]{\heiti 题{\xiaowu(中、英文)}目}\xddashuline[12cm]{\boldsong\songti\xd@timuI} \vspace{-3mm}

\noindent\makebox[3.2cm][s]{}\xddashuline[12cm]{\boldsong\songti\xd@timuII} \vspace{-3mm}

\noindent\makebox[3.2cm][s]{}\xddashuline[12cm]{\boldsong\xd@timuIII} \vspace{-3mm}

\noindent\makebox[3.2cm][s]{}\xddashuline[12cm]{\boldsong\xd@timuIIII} 

\noindent\makebox[3.2cm][s]{\heiti 作者姓名}\xddashuline[3cm]{\songti\sanhao\xd@xingming}
	\makebox[4.4cm][s]{\heiti 指导教师姓名、职务}\xddashuline[4.46cm]{\songti\sanhao\xd@jiaoshi}\vspace{-3mm}%\vspace{5mm}
	
\noindent\makebox[3.2cm][s]{\heiti 学科门类}\xddashuline[3cm]{\songti\sanhao\xd@menglei}
	\makebox[2.5cm][s]{\heiti 学科、专业}\xddashuline[6.3cm]{\songti\sanhao\xd@zhuanye}\vspace{-1mm}%\vspace{7mm}
	
\noindent\makebox[3.2cm][s]{\heiti 提交论文日期}\xddashuline[12cm]{\songti\sanhao\xd@riqi}
\makebox[1cm]{}

}

\cleardoublepage
\thispagestyle{empty}
{
%\baselinestretch{1.2}
%\setlength{\baselineskip}{26pt}
\begin{center}
\bfseries \sanhao
西安电子科技大学

学位论文独创性(或创新性)声明
\end{center}

\xiaosi \setlength{\baselineskip}{20pt}
秉承学校严谨的学风和优良的科学道德,本人声明所呈交的论文是我个人在导师指导下进行的研究工作及取得的研究成果。尽我所知,除了文中特别加以标注和致谢中所罗列的内容以外,论文中不包含其他人已经发表或撰写过的研究成果;也不包含为获得西安电子科技大学或其它教育机构的学位或证书而使用过的材料。与我一同工作的同志对本研究所做的任何贡献均已在论文中做了明确的说明并表示了谢意。

申请学位论文与资料若有不实之处,本人承担一切的法律责任。\\

本人签名:\xduline[3cm]{} \hspace{2.5cm} 日期:\xduline[3cm]{} \\     

\vspace{2cm}
\begin{center}
\bfseries \sanhao
西安电子科技大学

关于论文使用授权的说明
\end{center}

\xiaosi \setlength{\baselineskip}{20pt}
本人完全了解西安电子科技大学有关保留和使用学位论文的规定,即:研究生在校攻读学位期间论文工作的知识产权单位属西安电子科技大学。学校有权保留送交论文的复印件,允许查阅和借阅论文;学校可以公布论文的全部或部分内容,可以允许采用影印、缩印或其它复制手段保存论文。同时本人保证,毕业后结合学位论文研究课题再撰写的文章一律署名单位为西安电子科技大学。

本人授权西安电子科技大学图书馆保存学位论文,本学位论文属于\xd@miji(保密级别),在
\xduline[1cm]{}年解密后适用本授权书,并同意将论文在互联网上发布。\\\\

本人签名:\xduline[3cm]{} \hspace{2.5cm} 日期:\xduline[3cm]{}  \\      

导师签名:\xduline[3cm]{} \hspace{2.5cm} 日期:\xduline[3cm]{} 
}
}
\fi
}

%</cls>
%    \end{macrocode}
% \end{macro}
%
% \subsubsection{中文标题定义}
% \label{sec:theor}
%
% \changes{v0.2}{2009/06/06}{加入中文标题的定义。}
%
%    \begin{macrocode}
%<*cfg>
\renewcommand\contentsname{目\hspace{1em}录}
\renewcommand\listfigurename{插图索引}
\renewcommand\listtablename{表格索引}
\newcommand\listequationname{公式索引}
\newcommand\equationname{公式}
\renewcommand\bibname{参考文献}
\renewcommand\indexname{索引}
\renewcommand\figurename{图}
\renewcommand\tablename{表}
\newcommand\CJKprepartname{第}
\newcommand\CJKpartname{部分}
\newcommand\CJKthepart{\CJKnumber{\@arabic\c@part}}
\def\xd@CJKnumber#1{\ifcase#1{零}\or%
                    {一}\or{二}\or{三}\or{四}\or{五}\or%
                    {六}\or{七}\or{八}\or{九}\or{十}\or%
                    {十一}\or{十二}\or{十三}\or{十四}\or{十五}\or%
                    {十六}\or{十七}\or{十八}\or{十九}\or{二十}\fi}
\newcommand\CJKprechaptername{第}
\newcommand\CJKchaptername{章}
\newcommand\CJKthechapter{\xd@CJKnumber{\@arabic\c@chapter}}
\newcommand{\CJKthechaptername}[1]{%
            \CJKprechaptername\xd@CJKnumber{\@arabic#1}\CJKchaptername}
\renewcommand\chaptername{\CJKprechaptername\CJKthechapter\CJKchaptername}
\newcommand{\cabstractname}{摘\hspace{1em}要}
\newcommand{\eabstractname}{ABSTRACT}
\newcommand{\xd@ackname}{致\hspace{1em}谢}
\newcommand{\xd@ckeywords@title}{关键词:}
%</cfg>
%    \end{macrocode}

% \subsubsection{章节标题}
% \label{sec:titleandtoc}
%    \begin{macrocode}
%<*cls>
\titleformat{\chapter}[block]%
            {\sanhao\hei}{\chaptername}%
            {1ex}{\sanhao\hei\filcenter}%
            [\if@mainmatter\thispagestyle{xd@headings}\else\thispagestyle{xd@front}\fi]
\titlespacing*{\chapter}{0pt}{4ex}{3ex}[0pt]
%    \end{macrocode}
% \begin{macro}{\section}
% 一级节标题,例如:2.1  实验装置与实验方法
% 节标题序号与标题名之间空一个汉字符(下同)。
% 采用宋体四号(14pt)字居中书写。
%    \begin{macrocode}
\titleformat{\section}[block]%
            {\sihao[1.429]}{\thesection}%
            {1ex}{\sihao[1.429]\filcenter}
%    \end{macrocode}
% \end{macro}
%
% \begin{macro}{\subsection}
% 二级节标题,例如:2.1.1  实验方法
% 节标题序号与标题名之间空一个汉字符(下同)。
% 采用宋体小四号(12pt)字居左书写。
%    \begin{macrocode}
\titleformat{\subsection}[block]%
            {\xiaosi}{\thesubsection}%
            {1ex}{\xiaosi}
%</cls>
%    \end{macrocode}
% \end{macro}
%
%
%\subsubsection{目录格式}
%\label{sec:tableofcontents}
%
%    \begin{macrocode}
%<*cls>
\let\xd@orgtoc\tableofcontents
\renewcommand\tableofcontents{\xd@orgtoc\markright{\xd@title}}
\titlecontents{chapter}[0pt]{}%
              {\xiaosi\bfseries\song%
               \CJKthechaptername\thecontentslabel\enspace\enspace}%
              {\thecontentslabel}%
              {\titlerule*[.5pc]{.}\contentspage}
%</cls>
%    \end{macrocode}
%
%\subsubsection{数学相关}
%\label{sec:maths}
%
%    \begin{macrocode}
%<*cls>
\renewcommand\theequation{\ifnum \c@chapter>\z@ \thechapter%
                          -\fi\@arabic\c@equation}
%</cls>
%    \end{macrocode}
%
% \subsubsection{浮动对象以及表格}
% \label{sec:float}
%
% 设置浮动对象和文字之间的距离
% \changes{v0.2}{2009/06/06}{增加\cs{subfloat}}
%    \begin{macrocode}
%<*cls>
\let\old@tabular\@tabular
\def\xd@tabular{\dawu[1.5]\old@tabular}
\DeclareCaptionLabelFormat{xd@cap}{{\dawu[1.5] #1\rmfamily #2}}
\DeclareCaptionLabelSeparator{xd@sep}{\hspace{1em}}
\DeclareCaptionFont{xd@capfont}{\dawu[1.5]}
\captionsetup{labelformat=xd@cap,labelsep=xd@sep,font=xd@capfont}
\captionsetup[table]{position=top,belowskip={12bp-\intextsep},aboveskip=3bp} 
\captionsetup[figure]{position=bottom,belowskip={12bp-\intextsep},aboveskip=3bp}
\captionsetup[subfloat]{font=xd@capfont,captionskip=6bp,%
                        nearskip=6bp,farskip=0bp,topadjust=0bp}
		
%\AtBeginEnvironment{tabular}{\dawu[1.5]}		
\renewenvironment{table}{%
  \renewcommand* {\@floatboxreset}{%
    \reset@font\@setminipage}
  \dawu\@float{table}%
}{%
  \end@float\normalsize
}

 % \renewcommand{\thesubfigure}{\thefigure--(\arabic{subfigure})}
 % \renewcommand{\p@subfigure}{:}
%</cls>
%    \end{macrocode}

% \subsubsection{摘要格式}
% \label{sec:abstractformat}
%
% \begin{macro}{\xd@makeabstract}
% 中文摘要部分的标题为"摘要",用黑体三号字。
%    \begin{macrocode}
%<*cls>
\long\@xp\def\@xp\collect@@body\@xp#\@xp1\@xp\end\@xp#\@xp2\@xp{%
  \collect@@body{#1}\end{#2}}
\long\@xp\def\@xp\push@begins\@xp#\@xp1\@xp\begin\@xp#\@xp2\@xp{%
  \push@begins{#1}\begin{#2}}
\long\@xp\def\@xp\addto@envbody\@xp#\@xp1\@xp{%
  \addto@envbody{#1}}
\newcommand{\xd@@cabstract}[1]{\long\gdef\xd@cabstract{#1}}
\newenvironment{cabstract}{\collect@body\xd@@cabstract}{}
\newcommand{\xd@@eabstract}[1]{\long\gdef\xd@eabstract{#1}}
\newenvironment{eabstract}{\collect@body\xd@@eabstract}{}
\newcommand{\xd@@ckeywords}[1]{\long\gdef\xd@ckeywords{#1}}
\newenvironment{ckeywords}{\collect@body\xd@@ckeywords}{}
\newcommand{\xd@@ekeywords}[1]{\long\gdef\xd@ekeywords{#1}}
\newenvironment{ekeywords}{\collect@body\xd@@ekeywords}{}
\newcommand{\xd@makeabstract}{%
  \xd@mkabstracttrue%
% 摘要内容用小四号字书写,两端对齐,汉字用宋体,外文字用Times New Roman 体,标点
% 符号一律用中文输入状态下的标点符号
  \chapter*{\cabstractname}%
  \markboth{\cabstractname}{\xd@title}%
  \normalsize\xd@cabstract\vskip12bp%
% 关键词为悬挂缩进
  \setbox0=\hbox{\hei\xd@ckeywords@title\hspace{1em}}%
  \ifxd@bachelor\noindent\hangindent\wd0\hangafter1\fi
  \ifxd@master\noindent\hangindent\wd0\hangafter1\fi
  \box0{\hei\xd@ckeywords}%
% 英文摘要部分的标题为“ABSTRACT”,用Times New Roman 体三号字。
  \chapter*{\bfseries\eabstractname}%
  \markboth{\eabstractname}{\xd@title}%
  \normalsize\xd@eabstract\vskip12bp%
% 关键词为悬挂缩进
  \setbox0=\hbox{\bfseries Keywords:\hspace{1em}}%
  \ifxd@bachelor\noindent\hangindent\wd0\hangafter1\fi
  \ifxd@master\noindent\hangindent\wd0\hangafter1\fi
  \box0{\bfseries\xd@ekeywords}%
  \xd@mkabstractfalse%
}
%</cls>
%    \end{macrocode}
% \end{macro}
%
% \begin{macro}{\makecover}
%    \begin{macrocode}
%<cls>\newcommand\makecover{\xd@makeabstract}
%    \end{macrocode}
% \end{macro}
%
% \subsubsection{致谢}
% \label{sec:ack}
%
%    \begin{macrocode}
%<*cls>
\newenvironment{acknowledgments}{%
  \xd@numberedfalse%
  \chapter*{\xd@ackname}%
  \addcontentsline{toc}{chapter}{\bfseries\song\xd@ackname}%
  \markboth{\xd@ackname}{\xd@title}%
  \xd@numberedtrue}%
  {}
%</cls>
%    \end{macrocode}
%
% \subsubsection{参考文献}
% \label{sec:ref}
%
% \begin{macro}{\onlinecite}
% 正文引用模式。依赖于\pkg{natbib}宏包,修改其中的命令。
%    \begin{macrocode}
%<*cls>
\bibpunct{[}{]}{,}{s}{}{,}
\renewcommand\NAT@citesuper[3]{\ifNAT@swa
\unskip\kern\p@\textsuperscript{\NAT@@open #1\NAT@@close}%
   \if*#3*\else\ (#3)\fi\else #1\fi\endgroup}
\DeclareRobustCommand\onlinecite{\@onlinecite}
\def\@onlinecite#1{\begingroup\let\@cite\NAT@citenum\citep{#1}\endgroup}
%    \end{macrocode}
% \end{macro}
%
% 参考文献的正文部分用五号字。
% 行距采用固定值16磅,段前空3磅,段后空0磅。
%
% \begin{environment}{thebibliography}
% 修改默认的thebibliography环境,增加一些调整代码。
%    \begin{macrocode}
\renewenvironment{thebibliography}[1]{%    
  \xd@numberedfalse
  \chapter*{\bibname}%
  \addcontentsline{toc}{chapter}{\bfseries\song\bibname}%
  \markboth{\bibname}{\xd@title}%
  % \xd@numberedtrue%
  \wuhao[1.5]
  \list{\@biblabel{\@arabic\c@enumiv}}%
  {\renewcommand{\makelabel}[1]{##1\hfill}
    \settowidth\labelwidth{1.1cm}
    \setlength{\labelsep}{0.6em}
    \setlength{\itemindent}{0pt}
    \setlength{\leftmargin}{\labelwidth+\labelsep}
    \addtolength{\itemsep}{-0.7em}
    \usecounter{enumiv}%
    \let\p@enumiv\@empty
    \renewcommand\theenumiv{\@arabic\c@enumiv}}%
  \sloppy
  \clubpenalty4000
  \@clubpenalty \clubpenalty
  \widowpenalty4000%
  \interlinepenalty4000%
  \sfcode`\.\@m}
{\def\@noitemerr
  {\@latex@warning{Empty `thebibliography' environment}}%
  \endlist}
%</cls>
%    \end{macrocode}
% \end{environment}
%
%    \begin{macrocode}
%<cls>\AtEndOfClass{%% \iffalse
%%  Local Variables:
%%  mode: doctex
%%  TeX-master: t
%%  End:
%% \fi
%% \iffalse meta-comment
%%
%% Copyright (C) 2008 by Fei Qi <fred.qi@gmail.com>
%%
%% This file may be distributed and/or modified under the
%% conditions of the LaTeX Project Public License, either version 1.3a
%% of this license or (at your option) any later version.
%% The latest version of this license is in:
%%
%% http://www.latex-project.org/lppl.txt
%%
%% and version 1.3a or later is part of all distributions of LaTeX
%% version 2004/10/01 or later.
%%
%% \fi
%
% \CheckSum{714}
%
% \iffalse
%<*driver>
\ProvidesFile{xdthesis.dtx}[2009/06/15 0.9.1 Xidian University Thesis Template]
\documentclass[10pt]{ltxdoc}
\usepackage{url}
\usepackage{zhspacing,xltxtra}
\newfontfamily\zhfont[BoldFont={Adobe Heiti Std}]{Adobe Song Std}
\newfontfamily\zhpunctfont[BoldFont={Adobe Heiti Std}]{Adobe Song Std}
\usepackage{indentfirst}
\setlength{\parskip}{4pt plus1pt minus0pt}
\setlength{\topsep}{0pt}
\setlength{\partopsep}{0pt}
\setlength{\parindent}{20pt}
\addtolength{\oddsidemargin}{-1cm}
\advance\textwidth 1.5cm
\addtolength{\topmargin}{-1cm}
\addtolength{\headsep}{0.3cm}
\addtolength{\textheight}{2.3cm}
\renewcommand{\baselinestretch}{1.3}
% \DefineVerbatimEnvironment{example}{Verbatim}%
%   {frame=single, framerule=0.3mm, rulecolor=\color{red!75!green!50!blue}, 
%    fillcolor=\color{red!75!green!50!blue!15},framesep=2mm, baselinestretch=1.2,
%    fontsize=\small, gobble=1}
% \DefineVerbatimEnvironment{shell}{Verbatim}%
%   {frame=single, framerule=0.3mm, rulecolor=\color{red!85!green!60}, 
%    fillcolor=\color{red!85!green!10},framesep=2mm,fontsize=\small, gobble=1}
\makeatletter
\def\DescribeOption#1{\SpecialOptionIndex{#1}}
\def\tableofcontents{\renewcommand{\baselinestretch}{1.0}\@starttoc{toc}}
\def\DescribeMacro{\Describe@Macro}
\def\Describe@Macro#1{\PrintDescribeMacro{#1}\SpecialUsageIndex{#1}}
\def\PrintDescribeMacro#1{{\color{-red!75!green!50!blue!55}\MacroFont \string #1\hskip1em }}
\def\ps@headings{%
  \let\@oddfoot\@empty
  \def\@oddhead{\vbox{\hbox
    to\textwidth{\llap{\fbox{\rightmark\rule[-2pt]{0pt}{13pt}}}\hfil\thepage}\vskip-0.7pt
      \hbox to \textwidth{\hrulefill}}}%
  \let\@evenfoot\@oddfoot
  \let\@evenhead\@oddhead
  \let\@mkboth\markboth
  \def\sectionmark##1{%
    \markright{\ifnum \c@secnumdepth >\m@ne
      \thesection\quad
      \fi
      ##1}}
  \def\subsectionmark##1{%
    \markright{\ifnum \c@secnumdepth >\m@ne
      \thesubsection\quad
      \fi
      ##1}}
  \def\subsubsectionmark##1{%
    \markright{\ifnum \c@secnumdepth >\m@ne
      \thesubsubsection\quad
      \fi
      ##1}}}
\renewcommand\section{\@startsection {section}{1}{\z@}%
                                   {-3.5ex \@plus -1ex \@minus -.2ex}%
                                   {2.3ex \@plus.2ex}%
                                   {\normalfont\Large\bfseries}}
\renewcommand\subsection{\@startsection{subsection}{2}{\z@}%
                                     {-3.25ex\@plus -1ex \@minus -.2ex}%
                                     {1.5ex \@plus .2ex}%
                                     {\normalfont\large\bfseries}}
\renewcommand\subsubsection{\@startsection{subsubsection}{3}{\z@}%
                                     {-3.25ex\@plus -1ex \@minus -.2ex}%
                                     {1.5ex \@plus .2ex}%
                                     {\normalfont\normalsize\bfseries}}
\renewcommand\paragraph{\@startsection{paragraph}{4}{\z@}%
                                    {3.25ex \@plus1ex \@minus.2ex}%
                                    {-1em}%
                                    {\normalfont\normalsize\bfseries}}
\renewcommand\subparagraph{\@startsection{subparagraph}{5}{\parindent}%
                                       {3.25ex \@plus1ex \@minus .2ex}%
                                       {-1em}%
                                      {\normalfont\normalsize\bfseries}}
\makeatother
\pagestyle{empty}
\zhspacing
\EnableCrossrefs
\CodelineIndex
\RecordChanges
\begin{document}
\DocInput{xdthesis.dtx}
\clearpage
\end{document}
%</driver>
% \fi
%
% \GetFileInfo{xdthesis.dtx}
% \MakeShortVerb{\|}
%
% \changes{v0.9.1}{2009/06/15}{Template for Bachelor's thesis release candidate.}
% \changes{v0.9}{2009/06/10}{Template realy works for Bachelor's thesis.}
% \changes{v0.2}{2009/06/06}{Defined styles of fonts, names, and titles and
%   floats formats.}
% \changes{v0.1}{2008/06/24}{The \XeLaTeX\ template for writing thesis of Xidian
%   starts.}
%
% \def\xdthesis{\textsc{XD}\-\textsc{Thesis}}
% \def\pkg#1{\texttt{#1}}
% \def\xdu{西安电子科技大学}
%
% \DoNotIndex{\begin,\end,\begingroup,\endgroup}
% \DoNotIndex{\ifx,\ifdim,\ifnum,\ifcase,\else,\or,\fi}
% \DoNotIndex{\let,\def,\xdef,\newcommand,\renewcommand}
% \DoNotIndex{\expandafter,\csname,\endcsname,\relax,\protect}
% \DoNotIndex{\Huge,\huge,\LARGE,\Large,\large,\normalsize}
% \DoNotIndex{\small,\footnotesize,\scriptsize,\tiny}
% \DoNotIndex{\normalfont,\bfseries,\slshape,\interlinepenalty}
% \DoNotIndex{\hfil,\par,\hskip,\vskip,\vspace,\quad}
% \DoNotIndex{\centering,\raggedright}
% \DoNotIndex{\c@secnumdepth,\@startsection,\@setfontsize}
% \DoNotIndex{\ ,\@plus,\@minus,\p@,\z@,\@m,\@M,\@ne,\m@ne}
% \DoNotIndex{\@@par,\DeclareOperation,\RequirePackage,\LoadClass}
% \DoNotIndex{\AtBeginDocument,\AtEndDocument}
%
% \IndexPrologue{\section*{索引}%
%    \addcontentsline{toc}{section}{索~~~~引}}
% \GlossaryPrologue{\section*{修改记录}%
%    \addcontentsline{toc}{section}{修改记录}}
%
% \renewcommand{\abstractname}{摘~~要}
% \renewcommand{\contentsname}{目~~录}
%
% \title{\xdthesis:\xdu{}学位论文模板\thanks{Xidian University \XeLaTeX{}
%     Thesis Template.}}
% \author{齐飞\\[5pt]{\xdu{}电子工程学院}\\[5pt]
%   \texttt{fred.qi@gmail.com}} \date{v\fileversion\ (\filedate)}
% \maketitle\thispagestyle{empty}
%
% \begin{abstract}\noindent
%   此宏包旨在建立一个简单易用的西安电子科技大学学位论文模板,包括本科毕业设计、
%   硕士及博士学位论文。目前正在开发本科毕业设计的模板,对其它格式的支持会陆续加
%   入。
% \end{abstract}
%
% \vskip2cm 
% \def\abstractname{免责声明}
% \begin{abstract}
% \noindent
% \begin{enumerate}
% \item 本模板的发布遵守~\LaTeX{} Project Public License,使用前请认真阅读协议内容。
% \item 本模板为作者根据\xdu{}教务处颁发的《本科生毕业设计(论文)工作手册》编写
%   而成,旨在供\xdu{}毕业生撰写学位论文使用。
% \item 此模板仅为写作指南的参考实现,不保证格式审查老师不提意见。任何由于使用本
%   模板而引起的论文格式审查问题均与本模板作者无关。
% \item 任何个人或组织以本模板为基础进行修改、扩展而生成的新的专用模板,请严格遵
%   守~\LaTeX{} Project Public License 协议。由于违犯协议而引起的任何纠纷争端均与
%   本模板作者无关。
% \end{enumerate}
% \end{abstract}
%
% \clearpage
% \begin{multicols}{2}[
% \section*{\contentsname}
% \setlength{\columnseprule}{.4pt}
% \setlength{\columnsep}{18pt}]
%  \tableofcontents
% \end{multicols}
%
% \clearpage
% \pagenumbering{arabic}
% \pagestyle{headings}
% \section{模板介绍}
%
% \section{安装}
% \label{sec:installation}
%
% \subsection{下载}
% \xdthesis{} 的主页是:  \url{http://code.google.com/p/xdthesis/}。
%
%\section{致谢}
%\label{sec:thanks}
%
% \StopEventually{\PrintChanges\PrintIndex}
% \clearpage
%
% \section{实现细节}
%
% \subsection{基本信息}
%    \begin{macrocode}
%<cls>\NeedsTeXFormat{LaTeX2e}[1999/12/01]
%<cls>\ProvidesClass{xdthesis}
%<cfg>\ProvidesFile{xdthesis.cfg}
%<cls|cfg>[2009/06/06 0.2 Xidian University Thesis Template]
%    \end{macrocode}
%
% \subsection{定义选项}
% \label{sec:defoption}
%
% 定义论文类型以及是否涉密
%    \begin{macrocode}
%<*cls>
\hyphenation{XD-Thesis}
\def\xdthesis{\textsc{XDThesis}}
\def\version{0.1}
\newif\ifxd@bachelor\xd@bachelorfalse
\newif\ifxd@master\xd@masterfalse
\newif\ifxd@doctor\xd@doctorfalse
\newif\ifxd@secret\xd@secretfalse
\newif\ifxd@mkabstract\xd@mkabstractfalse
\newif\ifxd@numbered\xd@numberedtrue
\DeclareOption{bachelor}{\xd@bachelortrue}
\DeclareOption{master}{\xd@mastertrue}
\DeclareOption{doctor}{\xd@doctortrue}
\DeclareOption{secret}{\xd@secrettrue}
\AtEndOfClass{%
  \ifxd@bachelor\relax\else
    \ifxd@master\relax\else
      \ifxd@doctor\relax\else
        \ClassError{xdthesis}%
                   {Please specify a thesis option: bachelor, master or doctor.}{}
      \fi
    \fi
  \fi}
%    \end{macrocode}
%
%    \begin{macrocode}
\ExecuteOptions{arialtitle}
\ProcessOptions
\LoadClass[12pt, a4paper, openright]{book}
%</cls>
%    \end{macrocode}
%
% \subsection{装载宏包}
% \label{sec:loadpackage}
%
% 参考文献引用宏包。
%    \begin{macrocode}
%<*cls>
\RequirePackage[numbers,super,sort&compress]{natbib}
%    \end{macrocode}
%
%    \begin{macrocode}
% \RequirePackage{hyperref}
%    \end{macrocode}
% 引用的宏包和相应的定义。
% \pkg{hypernat} 让~\pkg{hyperref} 和~\pkg{natbib} 协调的工作。应该
% 在~\pkg{natbib} 和~\pkg{hyperref} 之后加载,参看其文档。
%    \begin{macrocode}
% \RequirePackage{hypernat}
%    \end{macrocode}
%
% 首行缩进。
%    \begin{macrocode}
\RequirePackage{indentfirst}
%    \end{macrocode}
%
% \changes{v0.9.1}{2009/06/15}{引用\pkg{paralist},缩小列表环境的行距。}
% 更好的列表环境。
%    \begin{macrocode}
\RequirePackage[neverdecrease]{paralist}
%    \end{macrocode}
%
% 页眉页脚。
%    \begin{macrocode}
% \RequirePackage{fancyhdr}
%    \end{macrocode}
%
% AMS\LaTeX{} 宏包,用来排出更加漂亮的公式
%    \begin{macrocode}
\RequirePackage{amsmath, amssymb}
%    \end{macrocode}
%
% 图形支持宏包。
%    \begin{macrocode}
\RequirePackage{graphicx}
%    \end{macrocode}
%
% 并排图形。\pkg{subfigure} 已经不再推荐,用新的~\pkg{subfig}。加入~|config| 选项以便兼容
% ~\pkg{subfigure} 的命令。
% 浮动图形和表格标题样式。\pkg{caption2} 已经不推荐使用,采用新的~\pkg{caption}。它会自动被
% ~\pkg{subfig} 装载进来。所以可以在后面看到~\cs{captionsetup} 的命令。
%    \begin{macrocode}
\RequirePackage{subfig}
%    \end{macrocode}
%
% 载入标题格式宏包。
%    \begin{macrocode}
\RequirePackage{ifthen}
\RequirePackage{titlesec,titletoc}
%    \end{macrocode}
% 本模板是基于\XeLaTeX{}的。
%    \begin{macrocode}
\RequirePackage{xunicode,xltxtra}
\RequirePackage[CJKnumber,CJKtextspaces,CJKmathspaces,BoldFont]{xeCJK}
\def\CJK@null{\kern\CJKnullspace\Unicode{48}{7}\kern\CJKnullspace}
\defaultfontfeatures{Mapping=tex-text} % after fontspec
\setCJKmainfont{Adobe Song Std}
\setCJKsansfont{Adobe Heiti Std}
% \setCJKmonofont{Adobe Kaiti Std}
\setCJKfamilyfont{song}{Adobe Song Std}
\setCJKfamilyfont{hei}{Adobe Heiti Std}
% \setCJKfamilyfont{fs}{Adobe Fangsong Std}
% \setCJKfamilyfont{kai}{Adobe Kaiti Std}
% \setCJKfamilyfont{li}{Adobe Kaiti Std} % todo: 用隶书字体代替
% \setCJKfamilyfont{you}{Adobe Kaiti Std} % todo: 用幼圆字体代替
\setmainfont{Times New Roman}
% \setsansfont{Arial}
\setmonofont{Courier Std}
%</cls>
%    \end{macrocode}
%
% \subsection{主文档格式}
% \label{sec:mainbody}
% \subsubsection{Three matters}
%
%    \begin{macrocode}
%<*cls>
\renewcommand\frontmatter{%
  \if@openright\cleardoublepage\else\clearpage\fi
  \@mainmatterfalse
  \pagenumbering{roman}
  \pagestyle{xd@headings}}
\renewcommand\mainmatter{%
  \cleardoublepage
  \@mainmattertrue
  \pagenumbering{arabic}
  \pagestyle{xd@headings}}
%</cls>
%    \end{macrocode}
%
% \subsubsection{字体}
% \label{sec:fonts}
% Ref 2:
% WORD 中的字号对应该关系如下:
% \begin{verbatim}
% 初号 = 42bp = 14.82mm = 42.1575pt
% 小初 = 36bp = 12.70mm = 36.135 pt
% 一号 = 26bp = 9.17mm = 26.0975pt
% 小一 = 24bp = 8.47mm = 24.09pt
% 二号 = 22bp = 7.76mm = 22.0825pt
% 小二 = 18bp = 6.35mm = 18.0675pt
% 三号 = 16bp = 5.64mm = 16.06pt
% 小三 = 15bp = 5.29mm = 15.05625pt
% 四号 = 14bp = 4.94mm = 14.0525pt
% 小四 = 12bp = 4.23mm = 12.045pt
% 五号 = 10.5bp = 3.70mm = 10.59375pt
% 小五 = 9bp = 3.18mm = 9.03375pt
% 六号 = 7.5bp = 2.56mm
% 小六 = 6.5bp = 2.29mm
% 七号 = 5.5bp = 1.94mm
% 八号 = 5bp = 1.76mm
%
% 1bp = 72.27/72 pt
% \end{verbatim}
%
%    \begin{macrocode}
%<*cls>
\newcommand{\song}{\CJKfamily{song}} % 宋体
\def\songti{\song}
\newcommand{\hei}{\CJKfamily{hei}} % 黑体
\def\heiti{\hei}
%    \end{macrocode}
%    \begin{macrocode}
\newlength\xd@linespace
\newcommand{\xd@choosefont}[2]{%
   \setlength{\xd@linespace}{#2*\real{#1}}%
   \fontsize{#2}{\xd@linespace}\selectfont}
\def\xd@define@fontsize#1#2{%
  \expandafter\newcommand\csname #1\endcsname[1][\baselinestretch]{%
    \xd@choosefont{##1}{#2}}}
\xd@define@fontsize{chuhao}{42bp}
\xd@define@fontsize{xiaochu}{36bp}
\xd@define@fontsize{yihao}{26bp}
\xd@define@fontsize{xiaoyi}{24bp}
\xd@define@fontsize{erhao}{22bp}
\xd@define@fontsize{xiaoer}{18bp}
\xd@define@fontsize{sanhao}{16bp}
\xd@define@fontsize{xiaosan}{15bp}
\xd@define@fontsize{sihao}{14bp}
\xd@define@fontsize{banxiaosi}{13bp}
\xd@define@fontsize{xiaosi}{12bp}
\xd@define@fontsize{dawu}{11bp}
\xd@define@fontsize{wuhao}{10.5bp}
\xd@define@fontsize{xiaowu}{9bp}
\xd@define@fontsize{liuhao}{7.5bp}
\xd@define@fontsize{xiaoliu}{6.5bp}
\xd@define@fontsize{qihao}{5.5bp}
\xd@define@fontsize{bahao}{5bp}
%    \end{macrocode}
%
% 定义行距,正文小四号(12pt)字,行距为1.5倍行距
%    \begin{macrocode}
\renewcommand\normalsize{\@setfontsize\normalsize{12bp}{18bp}}
\renewcommand\baselinestretch{1.3}
%</cls>
%    \end{macrocode}
%
% \subsubsection{页面设置}
% \label{sec:layout}
%
%    \begin{macrocode}
%<*cls>
\setlength{\textwidth}{\paperwidth}
\setlength{\textheight}{\paperheight}
\setlength\marginparwidth{0cm} 
\setlength\marginparsep{0cm}
\addtolength{\textwidth}{-6cm}
\setlength{\oddsidemargin}{4cm-1in}
\setlength{\evensidemargin}{2cm-1in}
\setlength{\topmargin}{1.45cm-1in} 
\setlength{\headheight}{20pt}
\setlength{\headsep}{0.6cm}
\setlength{\topskip}{0pt}
\setlength{\skip\footins}{15pt} 
\setlength{\footskip}{1.3cm}
\addtolength{\textheight}{-4.5cm}  
%</cls>
%    \end{macrocode}
%
% \subsubsection{页眉页脚}
% \label{sec:headerfooter}
%
% 新的一章最好从奇数页开始(openright),所以必须保证它前面那页如果没有内容也必
% 须没有页眉页脚。code stolen from fancyhdr
%    \begin{macrocode}
%<*cls>
\let\xd@cleardoublepage\cleardoublepage
\newcommand{\xd@clearemptydoublepage}{%
  \clearpage{\pagestyle{empty}\xd@cleardoublepage}}
\let\cleardoublepage\xd@clearemptydoublepage
\let\xd@orgtitle\title
\renewcommand{\title}[1]{\gdef\xd@title{#1}\xd@orgtitle{#1}}
%    \end{macrocode}
%
%
% 定义页眉和页脚。chapter 自动调用~thispagestyle{xd@abstract},所以要重新定
% 义~xd@abstract。
% \begin{macro}{\ps@xd@empty}
% \begin{macro}{\ps@xd@abstract}
% \begin{macro}{\ps@xd@headings}
% 定义页眉页脚格式:
% \begin{itemize}
% \item \texttt{xd@empty} :无页眉页脚
% \item \texttt{xd@abstract} :只显示页眉的页码
% \item \texttt{xd@headings}:页眉页脚同时显示
% \end{itemize}
%    \begin{macrocode}
\def\ps@xd@empty{%
  \let\@oddhead\@empty%
  \let\@evenhead\@empty%
  \let\@oddfoot\@empty%
  \let\@evenfoot\@empty}
\def\ps@xd@abstract{%
  \def\@oddhead{\vbox{\hbox to\textwidth{%
                \hfil{\wuhao\leftmark}\hfil}%
                \vskip2pt\rule{\textwidth}{0.75pt}}}%
  \def\@evenhead{\vbox{\hbox to\textwidth{%
                \hfil{\wuhao\rightmark}\hfil}%
                \vskip2pt\rule{\textwidth}{0.75pt}}}%
  \let\@oddfoot\@empty%
  \let\@evenfoot\@empty}
\def\ps@xd@headings{% 
  \def\@oddhead{\vbox{\hbox to\textwidth{%
                \hfil{\wuhao\leftmark}\hfil{\xiaowu\thepage}}%
                \vskip2pt\rule{\textwidth}{0.75pt}}}%
  \def\@evenhead{\vbox{\hbox to\textwidth{%
                {\xiaowu\thepage}\hfil{\wuhao\rightmark}\hfil}%
                \vskip2pt\rule{\textwidth}{0.75pt}}}%
  \let\@oddfoot\@empty%
  \let\@evenfoot\@empty}
%    \end{macrocode}
% \end{macro}
% \end{macro}
% \end{macro}
%
% 其实可以直接写到~\cs{chapter} 的定义里面。
%    \begin{macrocode} 
\renewcommand{\chaptermark}[1]{\markboth{\chaptername~~#1}{}}
\renewcommand{\sectionmark}[1]{\markright{\xd@title}}
%</cls>
%    \end{macrocode}
%
% \changes{v0.9.1}{2009/06/15}{增加首行按照两个中文字符缩进。}
% \subsubsection{段落}
% \label{sec:paragraph}
%
% 用于中文段落缩进和正文版式
%    \begin{macrocode}
%<*cls>
\newlength\CJK@twochars
\def\CJK@spaceChar{\Unicode{48}{7}}
\def\CJKindent{%
  \settowidth\CJK@twochars{\CJK@spaceChar\CJK@spaceChar}%
  \parindent\CJK@twochars}
%    \end{macrocode}
%
% 段落之间的竖直距离
%    \begin{macrocode}
\setlength{\parskip}{0pt \@plus2pt \@minus0pt}
%    \end{macrocode}
%
% 调整默认列表环境间的距离,以符合中文习惯。
% \begin{macro}{xd@item@space}
%    \begin{macrocode}
\def\xd@item@space{%
  \let\itemize\compactitem
  \let\enditemize\endcompactitem
  \let\enumerate\compactenum
  \let\endenumerate\endcompactenum
  \let\description\compactdesc
  \let\enddescription\endcompactdesc}
%</cls>
%    \end{macrocode}
% \end{macro}
%
% \subsubsection{中文标题定义}
% \label{sec:theor}
%
% \changes{v0.2}{2009/06/06}{加入中文标题的定义。}
%
%    \begin{macrocode}
%<*cfg>
\renewcommand\contentsname{目\hspace{1em}录}
\renewcommand\listfigurename{插图索引}
\renewcommand\listtablename{表格索引}
\newcommand\listequationname{公式索引}
\newcommand\equationname{公式}
\renewcommand\bibname{参考文献}
\renewcommand\indexname{索引}
\renewcommand\figurename{图}
\renewcommand\tablename{表}
\newcommand\CJKprepartname{第}
\newcommand\CJKpartname{部分}
\newcommand\CJKthepart{\CJKnumber{\@arabic\c@part}}
\def\xd@CJKnumber#1{\ifcase#1{零}\or%
                    {一}\or{二}\or{三}\or{四}\or{五}\or%
                    {六}\or{七}\or{八}\or{九}\or{十}\or%
                    {十一}\or{十二}\or{十三}\or{十四}\or{十五}\or%
                    {十六}\or{十七}\or{十八}\or{十九}\or{二十}\fi}
\newcommand\CJKprechaptername{第}
\newcommand\CJKchaptername{章}
\ifxd@bachelor
  \newcommand\CJKthechapter{\xd@CJKnumber{\@arabic\c@chapter}}
  \newcommand{\CJKthechaptername}[1]{%
              \CJKprechaptername~\xd@CJKnumber{\@arabic#1}~\CJKchaptername~~}
\fi
\renewcommand\chaptername{\CJKprechaptername~\CJKthechapter~\CJKchaptername}
\newcommand{\cabstractname}{摘\hspace{1em}要}
\newcommand{\eabstractname}{ABSTRACT}
\newcommand{\xd@ackname}{致\hspace{1em}谢}
\newcommand{\xd@ckeywords@title}{关键词:}
%</cfg>
%    \end{macrocode}

% \subsubsection{章节标题}
% \label{sec:titleandtoc}
%    \begin{macrocode}
%<*cls>
\titleformat{\chapter}[block]%
            {\sanhao\hei}{\chaptername}%
            {1ex}{\sanhao\hei\filcenter}%
            [\ifxd@mkabstract\thispagestyle{xd@abstract}\else\thispagestyle{xd@headings}\fi]
\titlespacing*{\chapter}{0pt}{4ex}{3ex}[0pt]
%    \end{macrocode}
% \begin{macro}{\section}
% 一级节标题,例如:2.1  实验装置与实验方法
% 节标题序号与标题名之间空一个汉字符(下同)。
% 采用宋体四号(14pt)字居中书写。
%    \begin{macrocode}
\titleformat{\section}[block]%
            {\sihao[1.429]}{\thesection}%
            {1ex}{\sihao[1.429]\filcenter}
%    \end{macrocode}
% \end{macro}
%
% \begin{macro}{\subsection}
% 二级节标题,例如:2.1.1  实验方法
% 节标题序号与标题名之间空一个汉字符(下同)。
% 采用宋体小四号(12pt)字居左书写。
%    \begin{macrocode}
\titleformat{\subsection}[block]%
            {\xiaosi}{\thesubsection}%
            {1ex}{\xiaosi}
%</cls>
%    \end{macrocode}
% \end{macro}
%
%
%\subsubsection{目录格式}
%\label{sec:tableofcontents}
%
%    \begin{macrocode}
%<*cls>
\let\xd@orgtoc\tableofcontents
\renewcommand\tableofcontents{\xd@orgtoc\markright{\contentsname}}
\titlecontents{chapter}[0pt]{}%
              {\xiaosi\bfseries\song%
                \ifxd@numbered\CJKthechaptername\thecontentslabel\else%
                \thecontentslabel\fi}{}%
              {\titlerule*[.6pc]{.}\contentspage}
%</cls>
%    \end{macrocode}
%
%\subsubsection{数学相关}
%\label{sec:maths}
%
%    \begin{macrocode}
%<*cls>
\renewcommand\theequation{\ifnum \c@chapter>\z@ \thechapter%
                          -\fi\@arabic\c@equation}
%</cls>
%    \end{macrocode}
%
% \subsubsection{浮动对象以及表格}
% \label{sec:float}
%
% 设置浮动对象和文字之间的距离
% \changes{v0.2}{2009/06/06}{增加~\cs{subfloat}}
%    \begin{macrocode}
%<*cls>
\let\old@tabular\@tabular
\def\xd@tabular{\dawu[1.5]\old@tabular}
\DeclareCaptionLabelFormat{xd@cap}{{\dawu[1.5] #1~\rmfamily #2}}
\DeclareCaptionLabelSeparator{xd@sep}{\hspace{1em}}
\DeclareCaptionFont{xd@capfont}{\dawu[1.5]}
\captionsetup{labelformat=xd@cap,labelsep=xd@sep,font=xd@capfont}
\captionsetup[table]{position=top,belowskip={12bp-\intextsep},aboveskip=3bp} 
\captionsetup[figure]{position=bottom,belowskip={12bp-\intextsep},aboveskip=3bp}
\captionsetup[subfloat]{font=xd@capfont,captionskip=6bp,%
                        nearskip=6bp,farskip=0bp,topadjust=0bp}
 % \renewcommand{\thesubfigure}{\thefigure--(\arabic{subfigure})}
 % \renewcommand{\p@subfigure}{:}
%</cls>
%    \end{macrocode}

% \subsubsection{摘要格式}
% \label{sec:abstractformat}
%
% \begin{macro}{\xd@makeabstract}
% 中文摘要部分的标题为"摘要",用黑体三号字。
%    \begin{macrocode}
%<*cls>
\long\@xp\def\@xp\collect@@body\@xp#\@xp1\@xp\end\@xp#\@xp2\@xp{%
  \collect@@body{#1}\end{#2}}
\long\@xp\def\@xp\push@begins\@xp#\@xp1\@xp\begin\@xp#\@xp2\@xp{%
  \push@begins{#1}\begin{#2}}
\long\@xp\def\@xp\addto@envbody\@xp#\@xp1\@xp{%
  \addto@envbody{#1}}
\newcommand{\xd@@cabstract}[1]{\long\gdef\xd@cabstract{#1}}
\newenvironment{cabstract}{\collect@body\xd@@cabstract}{}
\newcommand{\xd@@eabstract}[1]{\long\gdef\xd@eabstract{#1}}
\newenvironment{eabstract}{\collect@body\xd@@eabstract}{}
\newcommand{\xd@@ckeywords}[1]{\long\gdef\xd@ckeywords{#1}}
\newenvironment{ckeywords}{\collect@body\xd@@ckeywords}{}
\newcommand{\xd@@ekeywords}[1]{\long\gdef\xd@ekeywords{#1}}
\newenvironment{ekeywords}{\collect@body\xd@@ekeywords}{}
\newcommand{\xd@makeabstract}{%
  \xd@mkabstracttrue%
% 摘要内容用小四号字书写,两端对齐,汉字用宋体,外文字用~Times New Roman 体,标点
% 符号一律用中文输入状态下的标点符号
  \chapter*{\cabstractname}%
  \markboth{\cabstractname}{}%
  \normalsize\xd@cabstract\vskip12bp%
% 关键词为悬挂缩进
  \setbox0=\hbox{\hei\xd@ckeywords@title\hspace{1em}}%
  \ifxd@bachelor\noindent\hangindent\wd0\hangafter1\fi
  \box0{\hei\xd@ckeywords}%
% 英文摘要部分的标题为“ABSTRACT”,用~Times New Roman 体三号字。
  \chapter*{\bfseries\eabstractname}%
  \markboth{\eabstractname}{}%
  \normalsize\xd@eabstract\vskip12bp%
% 关键词为悬挂缩进
  \setbox0=\hbox{\bfseries Keywords:\hspace{1em}}%
  \ifxd@bachelor\noindent\hangindent\wd0\hangafter1\fi
  \box0{\bfseries\xd@ekeywords}%
  \xd@mkabstractfalse%
}
%</cls>
%    \end{macrocode}
% \end{macro}
%
% \begin{macro}{\makecover}
%    \begin{macrocode}
%<cls>\newcommand\makecover{\xd@makeabstract}
%    \end{macrocode}
% \end{macro}
%
% \subsubsection{致谢}
% \label{sec:ack}
%
%    \begin{macrocode}
%<*cls>
\newenvironment{ack}{%
  \xd@numberedfalse%
  \chapter*{\xd@ackname}%
  \addcontentsline{toc}{chapter}{\bfseries\song\xd@ackname}%
  \markboth{\xd@ackname}{}%
  \xd@numberedtrue}%
  {}
%</cls>
%    \end{macrocode}
%
% \subsubsection{参考文献}
% \label{sec:ref}
%
% \begin{macro}{\onlinecite}
% 正文引用模式。依赖于~\pkg{natbib}~宏包,修改其中的命令。
%    \begin{macrocode}
%<*cls>
\bibpunct{[}{]}{,}{s}{}{,}
\renewcommand\NAT@citesuper[3]{\ifNAT@swa
\unskip\kern\p@\textsuperscript{\NAT@@open #1\NAT@@close}%
   \if*#3*\else\ (#3)\fi\else #1\fi\endgroup}
\DeclareRobustCommand\onlinecite{\@onlinecite}
\def\@onlinecite#1{\begingroup\let\@cite\NAT@citenum\citep{#1}\endgroup}
%    \end{macrocode}
% \end{macro}
%
% 参考文献的正文部分用五号字。
% 行距采用固定值~16 磅,段前空~3 磅,段后空~0 磅。
%
% \begin{environment}{thebibliography}
% 修改默认的~thebibliography 环境,增加一些调整代码。
%    \begin{macrocode}
\renewenvironment{thebibliography}[1]{%    
  \xd@numberedfalse
  \chapter*{\bibname}%
  \addcontentsline{toc}{chapter}{\bfseries\song\bibname}%
  \markboth{\bibname}{\xd@title}%
  \xd@numberedtrue%
  \wuhao[1.5]
  \list{\@biblabel{\@arabic\c@enumiv}}%
  {\renewcommand{\makelabel}[1]{##1\hfill}
    \settowidth\labelwidth{1.1cm}
    \setlength{\labelsep}{0.6em}
    \setlength{\itemindent}{0pt}
    \setlength{\leftmargin}{\labelwidth+\labelsep}
    \addtolength{\itemsep}{-0.7em}
    \usecounter{enumiv}%
    \let\p@enumiv\@empty
    \renewcommand\theenumiv{\@arabic\c@enumiv}}%
  \sloppy
  \clubpenalty4000
  \@clubpenalty \clubpenalty
  \widowpenalty4000%
  \interlinepenalty4000%
  \sfcode`\.\@m}
{\def\@noitemerr
  {\@latex@warning{Empty `thebibliography' environment}}%
  \endlist}
%</cls>
%    \end{macrocode}
% \end{environment}
%
%    \begin{macrocode}
%<cls>\AtEndOfClass{%% \iffalse
%%  Local Variables:
%%  mode: doctex
%%  TeX-master: t
%%  End:
%% \fi
%% \iffalse meta-comment
%%
%% Copyright (C) 2008 by Fei Qi <fred.qi@gmail.com>
%%
%% This file may be distributed and/or modified under the
%% conditions of the LaTeX Project Public License, either version 1.3a
%% of this license or (at your option) any later version.
%% The latest version of this license is in:
%%
%% http://www.latex-project.org/lppl.txt
%%
%% and version 1.3a or later is part of all distributions of LaTeX
%% version 2004/10/01 or later.
%%
%% \fi
%
% \CheckSum{714}
%
% \iffalse
%<*driver>
\ProvidesFile{xdthesis.dtx}[2009/06/15 0.9.1 Xidian University Thesis Template]
\documentclass[10pt]{ltxdoc}
\usepackage{url}
\usepackage{zhspacing,xltxtra}
\newfontfamily\zhfont[BoldFont={Adobe Heiti Std}]{Adobe Song Std}
\newfontfamily\zhpunctfont[BoldFont={Adobe Heiti Std}]{Adobe Song Std}
\usepackage{indentfirst}
\setlength{\parskip}{4pt plus1pt minus0pt}
\setlength{\topsep}{0pt}
\setlength{\partopsep}{0pt}
\setlength{\parindent}{20pt}
\addtolength{\oddsidemargin}{-1cm}
\advance\textwidth 1.5cm
\addtolength{\topmargin}{-1cm}
\addtolength{\headsep}{0.3cm}
\addtolength{\textheight}{2.3cm}
\renewcommand{\baselinestretch}{1.3}
% \DefineVerbatimEnvironment{example}{Verbatim}%
%   {frame=single, framerule=0.3mm, rulecolor=\color{red!75!green!50!blue}, 
%    fillcolor=\color{red!75!green!50!blue!15},framesep=2mm, baselinestretch=1.2,
%    fontsize=\small, gobble=1}
% \DefineVerbatimEnvironment{shell}{Verbatim}%
%   {frame=single, framerule=0.3mm, rulecolor=\color{red!85!green!60}, 
%    fillcolor=\color{red!85!green!10},framesep=2mm,fontsize=\small, gobble=1}
\makeatletter
\def\DescribeOption#1{\SpecialOptionIndex{#1}}
\def\tableofcontents{\renewcommand{\baselinestretch}{1.0}\@starttoc{toc}}
\def\DescribeMacro{\Describe@Macro}
\def\Describe@Macro#1{\PrintDescribeMacro{#1}\SpecialUsageIndex{#1}}
\def\PrintDescribeMacro#1{{\color{-red!75!green!50!blue!55}\MacroFont \string #1\hskip1em }}
\def\ps@headings{%
  \let\@oddfoot\@empty
  \def\@oddhead{\vbox{\hbox
    to\textwidth{\llap{\fbox{\rightmark\rule[-2pt]{0pt}{13pt}}}\hfil\thepage}\vskip-0.7pt
      \hbox to \textwidth{\hrulefill}}}%
  \let\@evenfoot\@oddfoot
  \let\@evenhead\@oddhead
  \let\@mkboth\markboth
  \def\sectionmark##1{%
    \markright{\ifnum \c@secnumdepth >\m@ne
      \thesection\quad
      \fi
      ##1}}
  \def\subsectionmark##1{%
    \markright{\ifnum \c@secnumdepth >\m@ne
      \thesubsection\quad
      \fi
      ##1}}
  \def\subsubsectionmark##1{%
    \markright{\ifnum \c@secnumdepth >\m@ne
      \thesubsubsection\quad
      \fi
      ##1}}}
\renewcommand\section{\@startsection {section}{1}{\z@}%
                                   {-3.5ex \@plus -1ex \@minus -.2ex}%
                                   {2.3ex \@plus.2ex}%
                                   {\normalfont\Large\bfseries}}
\renewcommand\subsection{\@startsection{subsection}{2}{\z@}%
                                     {-3.25ex\@plus -1ex \@minus -.2ex}%
                                     {1.5ex \@plus .2ex}%
                                     {\normalfont\large\bfseries}}
\renewcommand\subsubsection{\@startsection{subsubsection}{3}{\z@}%
                                     {-3.25ex\@plus -1ex \@minus -.2ex}%
                                     {1.5ex \@plus .2ex}%
                                     {\normalfont\normalsize\bfseries}}
\renewcommand\paragraph{\@startsection{paragraph}{4}{\z@}%
                                    {3.25ex \@plus1ex \@minus.2ex}%
                                    {-1em}%
                                    {\normalfont\normalsize\bfseries}}
\renewcommand\subparagraph{\@startsection{subparagraph}{5}{\parindent}%
                                       {3.25ex \@plus1ex \@minus .2ex}%
                                       {-1em}%
                                      {\normalfont\normalsize\bfseries}}
\makeatother
\pagestyle{empty}
\zhspacing
\EnableCrossrefs
\CodelineIndex
\RecordChanges
\begin{document}
\DocInput{xdthesis.dtx}
\clearpage
\end{document}
%</driver>
% \fi
%
% \GetFileInfo{xdthesis.dtx}
% \MakeShortVerb{\|}
%
% \changes{v0.9.1}{2009/06/15}{Template for Bachelor's thesis release candidate.}
% \changes{v0.9}{2009/06/10}{Template realy works for Bachelor's thesis.}
% \changes{v0.2}{2009/06/06}{Defined styles of fonts, names, and titles and
%   floats formats.}
% \changes{v0.1}{2008/06/24}{The \XeLaTeX\ template for writing thesis of Xidian
%   starts.}
%
% \def\xdthesis{\textsc{XD}\-\textsc{Thesis}}
% \def\pkg#1{\texttt{#1}}
% \def\xdu{西安电子科技大学}
%
% \DoNotIndex{\begin,\end,\begingroup,\endgroup}
% \DoNotIndex{\ifx,\ifdim,\ifnum,\ifcase,\else,\or,\fi}
% \DoNotIndex{\let,\def,\xdef,\newcommand,\renewcommand}
% \DoNotIndex{\expandafter,\csname,\endcsname,\relax,\protect}
% \DoNotIndex{\Huge,\huge,\LARGE,\Large,\large,\normalsize}
% \DoNotIndex{\small,\footnotesize,\scriptsize,\tiny}
% \DoNotIndex{\normalfont,\bfseries,\slshape,\interlinepenalty}
% \DoNotIndex{\hfil,\par,\hskip,\vskip,\vspace,\quad}
% \DoNotIndex{\centering,\raggedright}
% \DoNotIndex{\c@secnumdepth,\@startsection,\@setfontsize}
% \DoNotIndex{\ ,\@plus,\@minus,\p@,\z@,\@m,\@M,\@ne,\m@ne}
% \DoNotIndex{\@@par,\DeclareOperation,\RequirePackage,\LoadClass}
% \DoNotIndex{\AtBeginDocument,\AtEndDocument}
%
% \IndexPrologue{\section*{索引}%
%    \addcontentsline{toc}{section}{索~~~~引}}
% \GlossaryPrologue{\section*{修改记录}%
%    \addcontentsline{toc}{section}{修改记录}}
%
% \renewcommand{\abstractname}{摘~~要}
% \renewcommand{\contentsname}{目~~录}
%
% \title{\xdthesis:\xdu{}学位论文模板\thanks{Xidian University \XeLaTeX{}
%     Thesis Template.}}
% \author{齐飞\\[5pt]{\xdu{}电子工程学院}\\[5pt]
%   \texttt{fred.qi@gmail.com}} \date{v\fileversion\ (\filedate)}
% \maketitle\thispagestyle{empty}
%
% \begin{abstract}\noindent
%   此宏包旨在建立一个简单易用的西安电子科技大学学位论文模板,包括本科毕业设计、
%   硕士及博士学位论文。目前正在开发本科毕业设计的模板,对其它格式的支持会陆续加
%   入。
% \end{abstract}
%
% \vskip2cm 
% \def\abstractname{免责声明}
% \begin{abstract}
% \noindent
% \begin{enumerate}
% \item 本模板的发布遵守~\LaTeX{} Project Public License,使用前请认真阅读协议内容。
% \item 本模板为作者根据\xdu{}教务处颁发的《本科生毕业设计(论文)工作手册》编写
%   而成,旨在供\xdu{}毕业生撰写学位论文使用。
% \item 此模板仅为写作指南的参考实现,不保证格式审查老师不提意见。任何由于使用本
%   模板而引起的论文格式审查问题均与本模板作者无关。
% \item 任何个人或组织以本模板为基础进行修改、扩展而生成的新的专用模板,请严格遵
%   守~\LaTeX{} Project Public License 协议。由于违犯协议而引起的任何纠纷争端均与
%   本模板作者无关。
% \end{enumerate}
% \end{abstract}
%
% \clearpage
% \begin{multicols}{2}[
% \section*{\contentsname}
% \setlength{\columnseprule}{.4pt}
% \setlength{\columnsep}{18pt}]
%  \tableofcontents
% \end{multicols}
%
% \clearpage
% \pagenumbering{arabic}
% \pagestyle{headings}
% \section{模板介绍}
%
% \section{安装}
% \label{sec:installation}
%
% \subsection{下载}
% \xdthesis{} 的主页是:  \url{http://code.google.com/p/xdthesis/}。
%
%\section{致谢}
%\label{sec:thanks}
%
% \StopEventually{\PrintChanges\PrintIndex}
% \clearpage
%
% \section{实现细节}
%
% \subsection{基本信息}
%    \begin{macrocode}
%<cls>\NeedsTeXFormat{LaTeX2e}[1999/12/01]
%<cls>\ProvidesClass{xdthesis}
%<cfg>\ProvidesFile{xdthesis.cfg}
%<cls|cfg>[2009/06/06 0.2 Xidian University Thesis Template]
%    \end{macrocode}
%
% \subsection{定义选项}
% \label{sec:defoption}
%
% 定义论文类型以及是否涉密
%    \begin{macrocode}
%<*cls>
\hyphenation{XD-Thesis}
\def\xdthesis{\textsc{XDThesis}}
\def\version{0.1}
\newif\ifxd@bachelor\xd@bachelorfalse
\newif\ifxd@master\xd@masterfalse
\newif\ifxd@doctor\xd@doctorfalse
\newif\ifxd@secret\xd@secretfalse
\newif\ifxd@mkabstract\xd@mkabstractfalse
\newif\ifxd@numbered\xd@numberedtrue
\DeclareOption{bachelor}{\xd@bachelortrue}
\DeclareOption{master}{\xd@mastertrue}
\DeclareOption{doctor}{\xd@doctortrue}
\DeclareOption{secret}{\xd@secrettrue}
\AtEndOfClass{%
  \ifxd@bachelor\relax\else
    \ifxd@master\relax\else
      \ifxd@doctor\relax\else
        \ClassError{xdthesis}%
                   {Please specify a thesis option: bachelor, master or doctor.}{}
      \fi
    \fi
  \fi}
%    \end{macrocode}
%
%    \begin{macrocode}
\ExecuteOptions{arialtitle}
\ProcessOptions
\LoadClass[12pt, a4paper, openright]{book}
%</cls>
%    \end{macrocode}
%
% \subsection{装载宏包}
% \label{sec:loadpackage}
%
% 参考文献引用宏包。
%    \begin{macrocode}
%<*cls>
\RequirePackage[numbers,super,sort&compress]{natbib}
%    \end{macrocode}
%
%    \begin{macrocode}
% \RequirePackage{hyperref}
%    \end{macrocode}
% 引用的宏包和相应的定义。
% \pkg{hypernat} 让~\pkg{hyperref} 和~\pkg{natbib} 协调的工作。应该
% 在~\pkg{natbib} 和~\pkg{hyperref} 之后加载,参看其文档。
%    \begin{macrocode}
% \RequirePackage{hypernat}
%    \end{macrocode}
%
% 首行缩进。
%    \begin{macrocode}
\RequirePackage{indentfirst}
%    \end{macrocode}
%
% \changes{v0.9.1}{2009/06/15}{引用\pkg{paralist},缩小列表环境的行距。}
% 更好的列表环境。
%    \begin{macrocode}
\RequirePackage[neverdecrease]{paralist}
%    \end{macrocode}
%
% 页眉页脚。
%    \begin{macrocode}
% \RequirePackage{fancyhdr}
%    \end{macrocode}
%
% AMS\LaTeX{} 宏包,用来排出更加漂亮的公式
%    \begin{macrocode}
\RequirePackage{amsmath, amssymb}
%    \end{macrocode}
%
% 图形支持宏包。
%    \begin{macrocode}
\RequirePackage{graphicx}
%    \end{macrocode}
%
% 并排图形。\pkg{subfigure} 已经不再推荐,用新的~\pkg{subfig}。加入~|config| 选项以便兼容
% ~\pkg{subfigure} 的命令。
% 浮动图形和表格标题样式。\pkg{caption2} 已经不推荐使用,采用新的~\pkg{caption}。它会自动被
% ~\pkg{subfig} 装载进来。所以可以在后面看到~\cs{captionsetup} 的命令。
%    \begin{macrocode}
\RequirePackage{subfig}
%    \end{macrocode}
%
% 载入标题格式宏包。
%    \begin{macrocode}
\RequirePackage{ifthen}
\RequirePackage{titlesec,titletoc}
%    \end{macrocode}
% 本模板是基于\XeLaTeX{}的。
%    \begin{macrocode}
\RequirePackage{xunicode,xltxtra}
\RequirePackage[CJKnumber,CJKtextspaces,CJKmathspaces,BoldFont]{xeCJK}
\def\CJK@null{\kern\CJKnullspace\Unicode{48}{7}\kern\CJKnullspace}
\defaultfontfeatures{Mapping=tex-text} % after fontspec
\setCJKmainfont{Adobe Song Std}
\setCJKsansfont{Adobe Heiti Std}
% \setCJKmonofont{Adobe Kaiti Std}
\setCJKfamilyfont{song}{Adobe Song Std}
\setCJKfamilyfont{hei}{Adobe Heiti Std}
% \setCJKfamilyfont{fs}{Adobe Fangsong Std}
% \setCJKfamilyfont{kai}{Adobe Kaiti Std}
% \setCJKfamilyfont{li}{Adobe Kaiti Std} % todo: 用隶书字体代替
% \setCJKfamilyfont{you}{Adobe Kaiti Std} % todo: 用幼圆字体代替
\setmainfont{Times New Roman}
% \setsansfont{Arial}
\setmonofont{Courier Std}
%</cls>
%    \end{macrocode}
%
% \subsection{主文档格式}
% \label{sec:mainbody}
% \subsubsection{Three matters}
%
%    \begin{macrocode}
%<*cls>
\renewcommand\frontmatter{%
  \if@openright\cleardoublepage\else\clearpage\fi
  \@mainmatterfalse
  \pagenumbering{roman}
  \pagestyle{xd@headings}}
\renewcommand\mainmatter{%
  \cleardoublepage
  \@mainmattertrue
  \pagenumbering{arabic}
  \pagestyle{xd@headings}}
%</cls>
%    \end{macrocode}
%
% \subsubsection{字体}
% \label{sec:fonts}
% Ref 2:
% WORD 中的字号对应该关系如下:
% \begin{verbatim}
% 初号 = 42bp = 14.82mm = 42.1575pt
% 小初 = 36bp = 12.70mm = 36.135 pt
% 一号 = 26bp = 9.17mm = 26.0975pt
% 小一 = 24bp = 8.47mm = 24.09pt
% 二号 = 22bp = 7.76mm = 22.0825pt
% 小二 = 18bp = 6.35mm = 18.0675pt
% 三号 = 16bp = 5.64mm = 16.06pt
% 小三 = 15bp = 5.29mm = 15.05625pt
% 四号 = 14bp = 4.94mm = 14.0525pt
% 小四 = 12bp = 4.23mm = 12.045pt
% 五号 = 10.5bp = 3.70mm = 10.59375pt
% 小五 = 9bp = 3.18mm = 9.03375pt
% 六号 = 7.5bp = 2.56mm
% 小六 = 6.5bp = 2.29mm
% 七号 = 5.5bp = 1.94mm
% 八号 = 5bp = 1.76mm
%
% 1bp = 72.27/72 pt
% \end{verbatim}
%
%    \begin{macrocode}
%<*cls>
\newcommand{\song}{\CJKfamily{song}} % 宋体
\def\songti{\song}
\newcommand{\hei}{\CJKfamily{hei}} % 黑体
\def\heiti{\hei}
%    \end{macrocode}
%    \begin{macrocode}
\newlength\xd@linespace
\newcommand{\xd@choosefont}[2]{%
   \setlength{\xd@linespace}{#2*\real{#1}}%
   \fontsize{#2}{\xd@linespace}\selectfont}
\def\xd@define@fontsize#1#2{%
  \expandafter\newcommand\csname #1\endcsname[1][\baselinestretch]{%
    \xd@choosefont{##1}{#2}}}
\xd@define@fontsize{chuhao}{42bp}
\xd@define@fontsize{xiaochu}{36bp}
\xd@define@fontsize{yihao}{26bp}
\xd@define@fontsize{xiaoyi}{24bp}
\xd@define@fontsize{erhao}{22bp}
\xd@define@fontsize{xiaoer}{18bp}
\xd@define@fontsize{sanhao}{16bp}
\xd@define@fontsize{xiaosan}{15bp}
\xd@define@fontsize{sihao}{14bp}
\xd@define@fontsize{banxiaosi}{13bp}
\xd@define@fontsize{xiaosi}{12bp}
\xd@define@fontsize{dawu}{11bp}
\xd@define@fontsize{wuhao}{10.5bp}
\xd@define@fontsize{xiaowu}{9bp}
\xd@define@fontsize{liuhao}{7.5bp}
\xd@define@fontsize{xiaoliu}{6.5bp}
\xd@define@fontsize{qihao}{5.5bp}
\xd@define@fontsize{bahao}{5bp}
%    \end{macrocode}
%
% 定义行距,正文小四号(12pt)字,行距为1.5倍行距
%    \begin{macrocode}
\renewcommand\normalsize{\@setfontsize\normalsize{12bp}{18bp}}
\renewcommand\baselinestretch{1.3}
%</cls>
%    \end{macrocode}
%
% \subsubsection{页面设置}
% \label{sec:layout}
%
%    \begin{macrocode}
%<*cls>
\setlength{\textwidth}{\paperwidth}
\setlength{\textheight}{\paperheight}
\setlength\marginparwidth{0cm} 
\setlength\marginparsep{0cm}
\addtolength{\textwidth}{-6cm}
\setlength{\oddsidemargin}{4cm-1in}
\setlength{\evensidemargin}{2cm-1in}
\setlength{\topmargin}{1.45cm-1in} 
\setlength{\headheight}{20pt}
\setlength{\headsep}{0.6cm}
\setlength{\topskip}{0pt}
\setlength{\skip\footins}{15pt} 
\setlength{\footskip}{1.3cm}
\addtolength{\textheight}{-4.5cm}  
%</cls>
%    \end{macrocode}
%
% \subsubsection{页眉页脚}
% \label{sec:headerfooter}
%
% 新的一章最好从奇数页开始(openright),所以必须保证它前面那页如果没有内容也必
% 须没有页眉页脚。code stolen from fancyhdr
%    \begin{macrocode}
%<*cls>
\let\xd@cleardoublepage\cleardoublepage
\newcommand{\xd@clearemptydoublepage}{%
  \clearpage{\pagestyle{empty}\xd@cleardoublepage}}
\let\cleardoublepage\xd@clearemptydoublepage
\let\xd@orgtitle\title
\renewcommand{\title}[1]{\gdef\xd@title{#1}\xd@orgtitle{#1}}
%    \end{macrocode}
%
%
% 定义页眉和页脚。chapter 自动调用~thispagestyle{xd@abstract},所以要重新定
% 义~xd@abstract。
% \begin{macro}{\ps@xd@empty}
% \begin{macro}{\ps@xd@abstract}
% \begin{macro}{\ps@xd@headings}
% 定义页眉页脚格式:
% \begin{itemize}
% \item \texttt{xd@empty} :无页眉页脚
% \item \texttt{xd@abstract} :只显示页眉的页码
% \item \texttt{xd@headings}:页眉页脚同时显示
% \end{itemize}
%    \begin{macrocode}
\def\ps@xd@empty{%
  \let\@oddhead\@empty%
  \let\@evenhead\@empty%
  \let\@oddfoot\@empty%
  \let\@evenfoot\@empty}
\def\ps@xd@abstract{%
  \def\@oddhead{\vbox{\hbox to\textwidth{%
                \hfil{\wuhao\leftmark}\hfil}%
                \vskip2pt\rule{\textwidth}{0.75pt}}}%
  \def\@evenhead{\vbox{\hbox to\textwidth{%
                \hfil{\wuhao\rightmark}\hfil}%
                \vskip2pt\rule{\textwidth}{0.75pt}}}%
  \let\@oddfoot\@empty%
  \let\@evenfoot\@empty}
\def\ps@xd@headings{% 
  \def\@oddhead{\vbox{\hbox to\textwidth{%
                \hfil{\wuhao\leftmark}\hfil{\xiaowu\thepage}}%
                \vskip2pt\rule{\textwidth}{0.75pt}}}%
  \def\@evenhead{\vbox{\hbox to\textwidth{%
                {\xiaowu\thepage}\hfil{\wuhao\rightmark}\hfil}%
                \vskip2pt\rule{\textwidth}{0.75pt}}}%
  \let\@oddfoot\@empty%
  \let\@evenfoot\@empty}
%    \end{macrocode}
% \end{macro}
% \end{macro}
% \end{macro}
%
% 其实可以直接写到~\cs{chapter} 的定义里面。
%    \begin{macrocode} 
\renewcommand{\chaptermark}[1]{\markboth{\chaptername~~#1}{}}
\renewcommand{\sectionmark}[1]{\markright{\xd@title}}
%</cls>
%    \end{macrocode}
%
% \changes{v0.9.1}{2009/06/15}{增加首行按照两个中文字符缩进。}
% \subsubsection{段落}
% \label{sec:paragraph}
%
% 用于中文段落缩进和正文版式
%    \begin{macrocode}
%<*cls>
\newlength\CJK@twochars
\def\CJK@spaceChar{\Unicode{48}{7}}
\def\CJKindent{%
  \settowidth\CJK@twochars{\CJK@spaceChar\CJK@spaceChar}%
  \parindent\CJK@twochars}
%    \end{macrocode}
%
% 段落之间的竖直距离
%    \begin{macrocode}
\setlength{\parskip}{0pt \@plus2pt \@minus0pt}
%    \end{macrocode}
%
% 调整默认列表环境间的距离,以符合中文习惯。
% \begin{macro}{xd@item@space}
%    \begin{macrocode}
\def\xd@item@space{%
  \let\itemize\compactitem
  \let\enditemize\endcompactitem
  \let\enumerate\compactenum
  \let\endenumerate\endcompactenum
  \let\description\compactdesc
  \let\enddescription\endcompactdesc}
%</cls>
%    \end{macrocode}
% \end{macro}
%
% \subsubsection{中文标题定义}
% \label{sec:theor}
%
% \changes{v0.2}{2009/06/06}{加入中文标题的定义。}
%
%    \begin{macrocode}
%<*cfg>
\renewcommand\contentsname{目\hspace{1em}录}
\renewcommand\listfigurename{插图索引}
\renewcommand\listtablename{表格索引}
\newcommand\listequationname{公式索引}
\newcommand\equationname{公式}
\renewcommand\bibname{参考文献}
\renewcommand\indexname{索引}
\renewcommand\figurename{图}
\renewcommand\tablename{表}
\newcommand\CJKprepartname{第}
\newcommand\CJKpartname{部分}
\newcommand\CJKthepart{\CJKnumber{\@arabic\c@part}}
\def\xd@CJKnumber#1{\ifcase#1{零}\or%
                    {一}\or{二}\or{三}\or{四}\or{五}\or%
                    {六}\or{七}\or{八}\or{九}\or{十}\or%
                    {十一}\or{十二}\or{十三}\or{十四}\or{十五}\or%
                    {十六}\or{十七}\or{十八}\or{十九}\or{二十}\fi}
\newcommand\CJKprechaptername{第}
\newcommand\CJKchaptername{章}
\ifxd@bachelor
  \newcommand\CJKthechapter{\xd@CJKnumber{\@arabic\c@chapter}}
  \newcommand{\CJKthechaptername}[1]{%
              \CJKprechaptername~\xd@CJKnumber{\@arabic#1}~\CJKchaptername~~}
\fi
\renewcommand\chaptername{\CJKprechaptername~\CJKthechapter~\CJKchaptername}
\newcommand{\cabstractname}{摘\hspace{1em}要}
\newcommand{\eabstractname}{ABSTRACT}
\newcommand{\xd@ackname}{致\hspace{1em}谢}
\newcommand{\xd@ckeywords@title}{关键词:}
%</cfg>
%    \end{macrocode}

% \subsubsection{章节标题}
% \label{sec:titleandtoc}
%    \begin{macrocode}
%<*cls>
\titleformat{\chapter}[block]%
            {\sanhao\hei}{\chaptername}%
            {1ex}{\sanhao\hei\filcenter}%
            [\ifxd@mkabstract\thispagestyle{xd@abstract}\else\thispagestyle{xd@headings}\fi]
\titlespacing*{\chapter}{0pt}{4ex}{3ex}[0pt]
%    \end{macrocode}
% \begin{macro}{\section}
% 一级节标题,例如:2.1  实验装置与实验方法
% 节标题序号与标题名之间空一个汉字符(下同)。
% 采用宋体四号(14pt)字居中书写。
%    \begin{macrocode}
\titleformat{\section}[block]%
            {\sihao[1.429]}{\thesection}%
            {1ex}{\sihao[1.429]\filcenter}
%    \end{macrocode}
% \end{macro}
%
% \begin{macro}{\subsection}
% 二级节标题,例如:2.1.1  实验方法
% 节标题序号与标题名之间空一个汉字符(下同)。
% 采用宋体小四号(12pt)字居左书写。
%    \begin{macrocode}
\titleformat{\subsection}[block]%
            {\xiaosi}{\thesubsection}%
            {1ex}{\xiaosi}
%</cls>
%    \end{macrocode}
% \end{macro}
%
%
%\subsubsection{目录格式}
%\label{sec:tableofcontents}
%
%    \begin{macrocode}
%<*cls>
\let\xd@orgtoc\tableofcontents
\renewcommand\tableofcontents{\xd@orgtoc\markright{\contentsname}}
\titlecontents{chapter}[0pt]{}%
              {\xiaosi\bfseries\song%
                \ifxd@numbered\CJKthechaptername\thecontentslabel\else%
                \thecontentslabel\fi}{}%
              {\titlerule*[.6pc]{.}\contentspage}
%</cls>
%    \end{macrocode}
%
%\subsubsection{数学相关}
%\label{sec:maths}
%
%    \begin{macrocode}
%<*cls>
\renewcommand\theequation{\ifnum \c@chapter>\z@ \thechapter%
                          -\fi\@arabic\c@equation}
%</cls>
%    \end{macrocode}
%
% \subsubsection{浮动对象以及表格}
% \label{sec:float}
%
% 设置浮动对象和文字之间的距离
% \changes{v0.2}{2009/06/06}{增加~\cs{subfloat}}
%    \begin{macrocode}
%<*cls>
\let\old@tabular\@tabular
\def\xd@tabular{\dawu[1.5]\old@tabular}
\DeclareCaptionLabelFormat{xd@cap}{{\dawu[1.5] #1~\rmfamily #2}}
\DeclareCaptionLabelSeparator{xd@sep}{\hspace{1em}}
\DeclareCaptionFont{xd@capfont}{\dawu[1.5]}
\captionsetup{labelformat=xd@cap,labelsep=xd@sep,font=xd@capfont}
\captionsetup[table]{position=top,belowskip={12bp-\intextsep},aboveskip=3bp} 
\captionsetup[figure]{position=bottom,belowskip={12bp-\intextsep},aboveskip=3bp}
\captionsetup[subfloat]{font=xd@capfont,captionskip=6bp,%
                        nearskip=6bp,farskip=0bp,topadjust=0bp}
 % \renewcommand{\thesubfigure}{\thefigure--(\arabic{subfigure})}
 % \renewcommand{\p@subfigure}{:}
%</cls>
%    \end{macrocode}

% \subsubsection{摘要格式}
% \label{sec:abstractformat}
%
% \begin{macro}{\xd@makeabstract}
% 中文摘要部分的标题为"摘要",用黑体三号字。
%    \begin{macrocode}
%<*cls>
\long\@xp\def\@xp\collect@@body\@xp#\@xp1\@xp\end\@xp#\@xp2\@xp{%
  \collect@@body{#1}\end{#2}}
\long\@xp\def\@xp\push@begins\@xp#\@xp1\@xp\begin\@xp#\@xp2\@xp{%
  \push@begins{#1}\begin{#2}}
\long\@xp\def\@xp\addto@envbody\@xp#\@xp1\@xp{%
  \addto@envbody{#1}}
\newcommand{\xd@@cabstract}[1]{\long\gdef\xd@cabstract{#1}}
\newenvironment{cabstract}{\collect@body\xd@@cabstract}{}
\newcommand{\xd@@eabstract}[1]{\long\gdef\xd@eabstract{#1}}
\newenvironment{eabstract}{\collect@body\xd@@eabstract}{}
\newcommand{\xd@@ckeywords}[1]{\long\gdef\xd@ckeywords{#1}}
\newenvironment{ckeywords}{\collect@body\xd@@ckeywords}{}
\newcommand{\xd@@ekeywords}[1]{\long\gdef\xd@ekeywords{#1}}
\newenvironment{ekeywords}{\collect@body\xd@@ekeywords}{}
\newcommand{\xd@makeabstract}{%
  \xd@mkabstracttrue%
% 摘要内容用小四号字书写,两端对齐,汉字用宋体,外文字用~Times New Roman 体,标点
% 符号一律用中文输入状态下的标点符号
  \chapter*{\cabstractname}%
  \markboth{\cabstractname}{}%
  \normalsize\xd@cabstract\vskip12bp%
% 关键词为悬挂缩进
  \setbox0=\hbox{\hei\xd@ckeywords@title\hspace{1em}}%
  \ifxd@bachelor\noindent\hangindent\wd0\hangafter1\fi
  \box0{\hei\xd@ckeywords}%
% 英文摘要部分的标题为“ABSTRACT”,用~Times New Roman 体三号字。
  \chapter*{\bfseries\eabstractname}%
  \markboth{\eabstractname}{}%
  \normalsize\xd@eabstract\vskip12bp%
% 关键词为悬挂缩进
  \setbox0=\hbox{\bfseries Keywords:\hspace{1em}}%
  \ifxd@bachelor\noindent\hangindent\wd0\hangafter1\fi
  \box0{\bfseries\xd@ekeywords}%
  \xd@mkabstractfalse%
}
%</cls>
%    \end{macrocode}
% \end{macro}
%
% \begin{macro}{\makecover}
%    \begin{macrocode}
%<cls>\newcommand\makecover{\xd@makeabstract}
%    \end{macrocode}
% \end{macro}
%
% \subsubsection{致谢}
% \label{sec:ack}
%
%    \begin{macrocode}
%<*cls>
\newenvironment{ack}{%
  \xd@numberedfalse%
  \chapter*{\xd@ackname}%
  \addcontentsline{toc}{chapter}{\bfseries\song\xd@ackname}%
  \markboth{\xd@ackname}{}%
  \xd@numberedtrue}%
  {}
%</cls>
%    \end{macrocode}
%
% \subsubsection{参考文献}
% \label{sec:ref}
%
% \begin{macro}{\onlinecite}
% 正文引用模式。依赖于~\pkg{natbib}~宏包,修改其中的命令。
%    \begin{macrocode}
%<*cls>
\bibpunct{[}{]}{,}{s}{}{,}
\renewcommand\NAT@citesuper[3]{\ifNAT@swa
\unskip\kern\p@\textsuperscript{\NAT@@open #1\NAT@@close}%
   \if*#3*\else\ (#3)\fi\else #1\fi\endgroup}
\DeclareRobustCommand\onlinecite{\@onlinecite}
\def\@onlinecite#1{\begingroup\let\@cite\NAT@citenum\citep{#1}\endgroup}
%    \end{macrocode}
% \end{macro}
%
% 参考文献的正文部分用五号字。
% 行距采用固定值~16 磅,段前空~3 磅,段后空~0 磅。
%
% \begin{environment}{thebibliography}
% 修改默认的~thebibliography 环境,增加一些调整代码。
%    \begin{macrocode}
\renewenvironment{thebibliography}[1]{%    
  \xd@numberedfalse
  \chapter*{\bibname}%
  \addcontentsline{toc}{chapter}{\bfseries\song\bibname}%
  \markboth{\bibname}{\xd@title}%
  \xd@numberedtrue%
  \wuhao[1.5]
  \list{\@biblabel{\@arabic\c@enumiv}}%
  {\renewcommand{\makelabel}[1]{##1\hfill}
    \settowidth\labelwidth{1.1cm}
    \setlength{\labelsep}{0.6em}
    \setlength{\itemindent}{0pt}
    \setlength{\leftmargin}{\labelwidth+\labelsep}
    \addtolength{\itemsep}{-0.7em}
    \usecounter{enumiv}%
    \let\p@enumiv\@empty
    \renewcommand\theenumiv{\@arabic\c@enumiv}}%
  \sloppy
  \clubpenalty4000
  \@clubpenalty \clubpenalty
  \widowpenalty4000%
  \interlinepenalty4000%
  \sfcode`\.\@m}
{\def\@noitemerr
  {\@latex@warning{Empty `thebibliography' environment}}%
  \endlist}
%</cls>
%    \end{macrocode}
% \end{environment}
%
%    \begin{macrocode}
%<cls>\AtEndOfClass{%% \iffalse
%%  Local Variables:
%%  mode: doctex
%%  TeX-master: t
%%  End:
%% \fi
%% \iffalse meta-comment
%%
%% Copyright (C) 2008 by Fei Qi <fred.qi@gmail.com>
%%
%% This file may be distributed and/or modified under the
%% conditions of the LaTeX Project Public License, either version 1.3a
%% of this license or (at your option) any later version.
%% The latest version of this license is in:
%%
%% http://www.latex-project.org/lppl.txt
%%
%% and version 1.3a or later is part of all distributions of LaTeX
%% version 2004/10/01 or later.
%%
%% \fi
%
% \CheckSum{714}
%
% \iffalse
%<*driver>
\ProvidesFile{xdthesis.dtx}[2009/06/15 0.9.1 Xidian University Thesis Template]
\documentclass[10pt]{ltxdoc}
\usepackage{url}
\usepackage{zhspacing,xltxtra}
\newfontfamily\zhfont[BoldFont={Adobe Heiti Std}]{Adobe Song Std}
\newfontfamily\zhpunctfont[BoldFont={Adobe Heiti Std}]{Adobe Song Std}
\usepackage{indentfirst}
\setlength{\parskip}{4pt plus1pt minus0pt}
\setlength{\topsep}{0pt}
\setlength{\partopsep}{0pt}
\setlength{\parindent}{20pt}
\addtolength{\oddsidemargin}{-1cm}
\advance\textwidth 1.5cm
\addtolength{\topmargin}{-1cm}
\addtolength{\headsep}{0.3cm}
\addtolength{\textheight}{2.3cm}
\renewcommand{\baselinestretch}{1.3}
% \DefineVerbatimEnvironment{example}{Verbatim}%
%   {frame=single, framerule=0.3mm, rulecolor=\color{red!75!green!50!blue}, 
%    fillcolor=\color{red!75!green!50!blue!15},framesep=2mm, baselinestretch=1.2,
%    fontsize=\small, gobble=1}
% \DefineVerbatimEnvironment{shell}{Verbatim}%
%   {frame=single, framerule=0.3mm, rulecolor=\color{red!85!green!60}, 
%    fillcolor=\color{red!85!green!10},framesep=2mm,fontsize=\small, gobble=1}
\makeatletter
\def\DescribeOption#1{\SpecialOptionIndex{#1}}
\def\tableofcontents{\renewcommand{\baselinestretch}{1.0}\@starttoc{toc}}
\def\DescribeMacro{\Describe@Macro}
\def\Describe@Macro#1{\PrintDescribeMacro{#1}\SpecialUsageIndex{#1}}
\def\PrintDescribeMacro#1{{\color{-red!75!green!50!blue!55}\MacroFont \string #1\hskip1em }}
\def\ps@headings{%
  \let\@oddfoot\@empty
  \def\@oddhead{\vbox{\hbox
    to\textwidth{\llap{\fbox{\rightmark\rule[-2pt]{0pt}{13pt}}}\hfil\thepage}\vskip-0.7pt
      \hbox to \textwidth{\hrulefill}}}%
  \let\@evenfoot\@oddfoot
  \let\@evenhead\@oddhead
  \let\@mkboth\markboth
  \def\sectionmark##1{%
    \markright{\ifnum \c@secnumdepth >\m@ne
      \thesection\quad
      \fi
      ##1}}
  \def\subsectionmark##1{%
    \markright{\ifnum \c@secnumdepth >\m@ne
      \thesubsection\quad
      \fi
      ##1}}
  \def\subsubsectionmark##1{%
    \markright{\ifnum \c@secnumdepth >\m@ne
      \thesubsubsection\quad
      \fi
      ##1}}}
\renewcommand\section{\@startsection {section}{1}{\z@}%
                                   {-3.5ex \@plus -1ex \@minus -.2ex}%
                                   {2.3ex \@plus.2ex}%
                                   {\normalfont\Large\bfseries}}
\renewcommand\subsection{\@startsection{subsection}{2}{\z@}%
                                     {-3.25ex\@plus -1ex \@minus -.2ex}%
                                     {1.5ex \@plus .2ex}%
                                     {\normalfont\large\bfseries}}
\renewcommand\subsubsection{\@startsection{subsubsection}{3}{\z@}%
                                     {-3.25ex\@plus -1ex \@minus -.2ex}%
                                     {1.5ex \@plus .2ex}%
                                     {\normalfont\normalsize\bfseries}}
\renewcommand\paragraph{\@startsection{paragraph}{4}{\z@}%
                                    {3.25ex \@plus1ex \@minus.2ex}%
                                    {-1em}%
                                    {\normalfont\normalsize\bfseries}}
\renewcommand\subparagraph{\@startsection{subparagraph}{5}{\parindent}%
                                       {3.25ex \@plus1ex \@minus .2ex}%
                                       {-1em}%
                                      {\normalfont\normalsize\bfseries}}
\makeatother
\pagestyle{empty}
\zhspacing
\EnableCrossrefs
\CodelineIndex
\RecordChanges
\begin{document}
\DocInput{xdthesis.dtx}
\clearpage
\end{document}
%</driver>
% \fi
%
% \GetFileInfo{xdthesis.dtx}
% \MakeShortVerb{\|}
%
% \changes{v0.9.1}{2009/06/15}{Template for Bachelor's thesis release candidate.}
% \changes{v0.9}{2009/06/10}{Template realy works for Bachelor's thesis.}
% \changes{v0.2}{2009/06/06}{Defined styles of fonts, names, and titles and
%   floats formats.}
% \changes{v0.1}{2008/06/24}{The \XeLaTeX\ template for writing thesis of Xidian
%   starts.}
%
% \def\xdthesis{\textsc{XD}\-\textsc{Thesis}}
% \def\pkg#1{\texttt{#1}}
% \def\xdu{西安电子科技大学}
%
% \DoNotIndex{\begin,\end,\begingroup,\endgroup}
% \DoNotIndex{\ifx,\ifdim,\ifnum,\ifcase,\else,\or,\fi}
% \DoNotIndex{\let,\def,\xdef,\newcommand,\renewcommand}
% \DoNotIndex{\expandafter,\csname,\endcsname,\relax,\protect}
% \DoNotIndex{\Huge,\huge,\LARGE,\Large,\large,\normalsize}
% \DoNotIndex{\small,\footnotesize,\scriptsize,\tiny}
% \DoNotIndex{\normalfont,\bfseries,\slshape,\interlinepenalty}
% \DoNotIndex{\hfil,\par,\hskip,\vskip,\vspace,\quad}
% \DoNotIndex{\centering,\raggedright}
% \DoNotIndex{\c@secnumdepth,\@startsection,\@setfontsize}
% \DoNotIndex{\ ,\@plus,\@minus,\p@,\z@,\@m,\@M,\@ne,\m@ne}
% \DoNotIndex{\@@par,\DeclareOperation,\RequirePackage,\LoadClass}
% \DoNotIndex{\AtBeginDocument,\AtEndDocument}
%
% \IndexPrologue{\section*{索引}%
%    \addcontentsline{toc}{section}{索~~~~引}}
% \GlossaryPrologue{\section*{修改记录}%
%    \addcontentsline{toc}{section}{修改记录}}
%
% \renewcommand{\abstractname}{摘~~要}
% \renewcommand{\contentsname}{目~~录}
%
% \title{\xdthesis:\xdu{}学位论文模板\thanks{Xidian University \XeLaTeX{}
%     Thesis Template.}}
% \author{齐飞\\[5pt]{\xdu{}电子工程学院}\\[5pt]
%   \texttt{fred.qi@gmail.com}} \date{v\fileversion\ (\filedate)}
% \maketitle\thispagestyle{empty}
%
% \begin{abstract}\noindent
%   此宏包旨在建立一个简单易用的西安电子科技大学学位论文模板,包括本科毕业设计、
%   硕士及博士学位论文。目前正在开发本科毕业设计的模板,对其它格式的支持会陆续加
%   入。
% \end{abstract}
%
% \vskip2cm 
% \def\abstractname{免责声明}
% \begin{abstract}
% \noindent
% \begin{enumerate}
% \item 本模板的发布遵守~\LaTeX{} Project Public License,使用前请认真阅读协议内容。
% \item 本模板为作者根据\xdu{}教务处颁发的《本科生毕业设计(论文)工作手册》编写
%   而成,旨在供\xdu{}毕业生撰写学位论文使用。
% \item 此模板仅为写作指南的参考实现,不保证格式审查老师不提意见。任何由于使用本
%   模板而引起的论文格式审查问题均与本模板作者无关。
% \item 任何个人或组织以本模板为基础进行修改、扩展而生成的新的专用模板,请严格遵
%   守~\LaTeX{} Project Public License 协议。由于违犯协议而引起的任何纠纷争端均与
%   本模板作者无关。
% \end{enumerate}
% \end{abstract}
%
% \clearpage
% \begin{multicols}{2}[
% \section*{\contentsname}
% \setlength{\columnseprule}{.4pt}
% \setlength{\columnsep}{18pt}]
%  \tableofcontents
% \end{multicols}
%
% \clearpage
% \pagenumbering{arabic}
% \pagestyle{headings}
% \section{模板介绍}
%
% \section{安装}
% \label{sec:installation}
%
% \subsection{下载}
% \xdthesis{} 的主页是:  \url{http://code.google.com/p/xdthesis/}。
%
%\section{致谢}
%\label{sec:thanks}
%
% \StopEventually{\PrintChanges\PrintIndex}
% \clearpage
%
% \section{实现细节}
%
% \subsection{基本信息}
%    \begin{macrocode}
%<cls>\NeedsTeXFormat{LaTeX2e}[1999/12/01]
%<cls>\ProvidesClass{xdthesis}
%<cfg>\ProvidesFile{xdthesis.cfg}
%<cls|cfg>[2009/06/06 0.2 Xidian University Thesis Template]
%    \end{macrocode}
%
% \subsection{定义选项}
% \label{sec:defoption}
%
% 定义论文类型以及是否涉密
%    \begin{macrocode}
%<*cls>
\hyphenation{XD-Thesis}
\def\xdthesis{\textsc{XDThesis}}
\def\version{0.1}
\newif\ifxd@bachelor\xd@bachelorfalse
\newif\ifxd@master\xd@masterfalse
\newif\ifxd@doctor\xd@doctorfalse
\newif\ifxd@secret\xd@secretfalse
\newif\ifxd@mkabstract\xd@mkabstractfalse
\newif\ifxd@numbered\xd@numberedtrue
\DeclareOption{bachelor}{\xd@bachelortrue}
\DeclareOption{master}{\xd@mastertrue}
\DeclareOption{doctor}{\xd@doctortrue}
\DeclareOption{secret}{\xd@secrettrue}
\AtEndOfClass{%
  \ifxd@bachelor\relax\else
    \ifxd@master\relax\else
      \ifxd@doctor\relax\else
        \ClassError{xdthesis}%
                   {Please specify a thesis option: bachelor, master or doctor.}{}
      \fi
    \fi
  \fi}
%    \end{macrocode}
%
%    \begin{macrocode}
\ExecuteOptions{arialtitle}
\ProcessOptions
\LoadClass[12pt, a4paper, openright]{book}
%</cls>
%    \end{macrocode}
%
% \subsection{装载宏包}
% \label{sec:loadpackage}
%
% 参考文献引用宏包。
%    \begin{macrocode}
%<*cls>
\RequirePackage[numbers,super,sort&compress]{natbib}
%    \end{macrocode}
%
%    \begin{macrocode}
% \RequirePackage{hyperref}
%    \end{macrocode}
% 引用的宏包和相应的定义。
% \pkg{hypernat} 让~\pkg{hyperref} 和~\pkg{natbib} 协调的工作。应该
% 在~\pkg{natbib} 和~\pkg{hyperref} 之后加载,参看其文档。
%    \begin{macrocode}
% \RequirePackage{hypernat}
%    \end{macrocode}
%
% 首行缩进。
%    \begin{macrocode}
\RequirePackage{indentfirst}
%    \end{macrocode}
%
% \changes{v0.9.1}{2009/06/15}{引用\pkg{paralist},缩小列表环境的行距。}
% 更好的列表环境。
%    \begin{macrocode}
\RequirePackage[neverdecrease]{paralist}
%    \end{macrocode}
%
% 页眉页脚。
%    \begin{macrocode}
% \RequirePackage{fancyhdr}
%    \end{macrocode}
%
% AMS\LaTeX{} 宏包,用来排出更加漂亮的公式
%    \begin{macrocode}
\RequirePackage{amsmath, amssymb}
%    \end{macrocode}
%
% 图形支持宏包。
%    \begin{macrocode}
\RequirePackage{graphicx}
%    \end{macrocode}
%
% 并排图形。\pkg{subfigure} 已经不再推荐,用新的~\pkg{subfig}。加入~|config| 选项以便兼容
% ~\pkg{subfigure} 的命令。
% 浮动图形和表格标题样式。\pkg{caption2} 已经不推荐使用,采用新的~\pkg{caption}。它会自动被
% ~\pkg{subfig} 装载进来。所以可以在后面看到~\cs{captionsetup} 的命令。
%    \begin{macrocode}
\RequirePackage{subfig}
%    \end{macrocode}
%
% 载入标题格式宏包。
%    \begin{macrocode}
\RequirePackage{ifthen}
\RequirePackage{titlesec,titletoc}
%    \end{macrocode}
% 本模板是基于\XeLaTeX{}的。
%    \begin{macrocode}
\RequirePackage{xunicode,xltxtra}
\RequirePackage[CJKnumber,CJKtextspaces,CJKmathspaces,BoldFont]{xeCJK}
\def\CJK@null{\kern\CJKnullspace\Unicode{48}{7}\kern\CJKnullspace}
\defaultfontfeatures{Mapping=tex-text} % after fontspec
\setCJKmainfont{Adobe Song Std}
\setCJKsansfont{Adobe Heiti Std}
% \setCJKmonofont{Adobe Kaiti Std}
\setCJKfamilyfont{song}{Adobe Song Std}
\setCJKfamilyfont{hei}{Adobe Heiti Std}
% \setCJKfamilyfont{fs}{Adobe Fangsong Std}
% \setCJKfamilyfont{kai}{Adobe Kaiti Std}
% \setCJKfamilyfont{li}{Adobe Kaiti Std} % todo: 用隶书字体代替
% \setCJKfamilyfont{you}{Adobe Kaiti Std} % todo: 用幼圆字体代替
\setmainfont{Times New Roman}
% \setsansfont{Arial}
\setmonofont{Courier Std}
%</cls>
%    \end{macrocode}
%
% \subsection{主文档格式}
% \label{sec:mainbody}
% \subsubsection{Three matters}
%
%    \begin{macrocode}
%<*cls>
\renewcommand\frontmatter{%
  \if@openright\cleardoublepage\else\clearpage\fi
  \@mainmatterfalse
  \pagenumbering{roman}
  \pagestyle{xd@headings}}
\renewcommand\mainmatter{%
  \cleardoublepage
  \@mainmattertrue
  \pagenumbering{arabic}
  \pagestyle{xd@headings}}
%</cls>
%    \end{macrocode}
%
% \subsubsection{字体}
% \label{sec:fonts}
% Ref 2:
% WORD 中的字号对应该关系如下:
% \begin{verbatim}
% 初号 = 42bp = 14.82mm = 42.1575pt
% 小初 = 36bp = 12.70mm = 36.135 pt
% 一号 = 26bp = 9.17mm = 26.0975pt
% 小一 = 24bp = 8.47mm = 24.09pt
% 二号 = 22bp = 7.76mm = 22.0825pt
% 小二 = 18bp = 6.35mm = 18.0675pt
% 三号 = 16bp = 5.64mm = 16.06pt
% 小三 = 15bp = 5.29mm = 15.05625pt
% 四号 = 14bp = 4.94mm = 14.0525pt
% 小四 = 12bp = 4.23mm = 12.045pt
% 五号 = 10.5bp = 3.70mm = 10.59375pt
% 小五 = 9bp = 3.18mm = 9.03375pt
% 六号 = 7.5bp = 2.56mm
% 小六 = 6.5bp = 2.29mm
% 七号 = 5.5bp = 1.94mm
% 八号 = 5bp = 1.76mm
%
% 1bp = 72.27/72 pt
% \end{verbatim}
%
%    \begin{macrocode}
%<*cls>
\newcommand{\song}{\CJKfamily{song}} % 宋体
\def\songti{\song}
\newcommand{\hei}{\CJKfamily{hei}} % 黑体
\def\heiti{\hei}
%    \end{macrocode}
%    \begin{macrocode}
\newlength\xd@linespace
\newcommand{\xd@choosefont}[2]{%
   \setlength{\xd@linespace}{#2*\real{#1}}%
   \fontsize{#2}{\xd@linespace}\selectfont}
\def\xd@define@fontsize#1#2{%
  \expandafter\newcommand\csname #1\endcsname[1][\baselinestretch]{%
    \xd@choosefont{##1}{#2}}}
\xd@define@fontsize{chuhao}{42bp}
\xd@define@fontsize{xiaochu}{36bp}
\xd@define@fontsize{yihao}{26bp}
\xd@define@fontsize{xiaoyi}{24bp}
\xd@define@fontsize{erhao}{22bp}
\xd@define@fontsize{xiaoer}{18bp}
\xd@define@fontsize{sanhao}{16bp}
\xd@define@fontsize{xiaosan}{15bp}
\xd@define@fontsize{sihao}{14bp}
\xd@define@fontsize{banxiaosi}{13bp}
\xd@define@fontsize{xiaosi}{12bp}
\xd@define@fontsize{dawu}{11bp}
\xd@define@fontsize{wuhao}{10.5bp}
\xd@define@fontsize{xiaowu}{9bp}
\xd@define@fontsize{liuhao}{7.5bp}
\xd@define@fontsize{xiaoliu}{6.5bp}
\xd@define@fontsize{qihao}{5.5bp}
\xd@define@fontsize{bahao}{5bp}
%    \end{macrocode}
%
% 定义行距,正文小四号(12pt)字,行距为1.5倍行距
%    \begin{macrocode}
\renewcommand\normalsize{\@setfontsize\normalsize{12bp}{18bp}}
\renewcommand\baselinestretch{1.3}
%</cls>
%    \end{macrocode}
%
% \subsubsection{页面设置}
% \label{sec:layout}
%
%    \begin{macrocode}
%<*cls>
\setlength{\textwidth}{\paperwidth}
\setlength{\textheight}{\paperheight}
\setlength\marginparwidth{0cm} 
\setlength\marginparsep{0cm}
\addtolength{\textwidth}{-6cm}
\setlength{\oddsidemargin}{4cm-1in}
\setlength{\evensidemargin}{2cm-1in}
\setlength{\topmargin}{1.45cm-1in} 
\setlength{\headheight}{20pt}
\setlength{\headsep}{0.6cm}
\setlength{\topskip}{0pt}
\setlength{\skip\footins}{15pt} 
\setlength{\footskip}{1.3cm}
\addtolength{\textheight}{-4.5cm}  
%</cls>
%    \end{macrocode}
%
% \subsubsection{页眉页脚}
% \label{sec:headerfooter}
%
% 新的一章最好从奇数页开始(openright),所以必须保证它前面那页如果没有内容也必
% 须没有页眉页脚。code stolen from fancyhdr
%    \begin{macrocode}
%<*cls>
\let\xd@cleardoublepage\cleardoublepage
\newcommand{\xd@clearemptydoublepage}{%
  \clearpage{\pagestyle{empty}\xd@cleardoublepage}}
\let\cleardoublepage\xd@clearemptydoublepage
\let\xd@orgtitle\title
\renewcommand{\title}[1]{\gdef\xd@title{#1}\xd@orgtitle{#1}}
%    \end{macrocode}
%
%
% 定义页眉和页脚。chapter 自动调用~thispagestyle{xd@abstract},所以要重新定
% 义~xd@abstract。
% \begin{macro}{\ps@xd@empty}
% \begin{macro}{\ps@xd@abstract}
% \begin{macro}{\ps@xd@headings}
% 定义页眉页脚格式:
% \begin{itemize}
% \item \texttt{xd@empty} :无页眉页脚
% \item \texttt{xd@abstract} :只显示页眉的页码
% \item \texttt{xd@headings}:页眉页脚同时显示
% \end{itemize}
%    \begin{macrocode}
\def\ps@xd@empty{%
  \let\@oddhead\@empty%
  \let\@evenhead\@empty%
  \let\@oddfoot\@empty%
  \let\@evenfoot\@empty}
\def\ps@xd@abstract{%
  \def\@oddhead{\vbox{\hbox to\textwidth{%
                \hfil{\wuhao\leftmark}\hfil}%
                \vskip2pt\rule{\textwidth}{0.75pt}}}%
  \def\@evenhead{\vbox{\hbox to\textwidth{%
                \hfil{\wuhao\rightmark}\hfil}%
                \vskip2pt\rule{\textwidth}{0.75pt}}}%
  \let\@oddfoot\@empty%
  \let\@evenfoot\@empty}
\def\ps@xd@headings{% 
  \def\@oddhead{\vbox{\hbox to\textwidth{%
                \hfil{\wuhao\leftmark}\hfil{\xiaowu\thepage}}%
                \vskip2pt\rule{\textwidth}{0.75pt}}}%
  \def\@evenhead{\vbox{\hbox to\textwidth{%
                {\xiaowu\thepage}\hfil{\wuhao\rightmark}\hfil}%
                \vskip2pt\rule{\textwidth}{0.75pt}}}%
  \let\@oddfoot\@empty%
  \let\@evenfoot\@empty}
%    \end{macrocode}
% \end{macro}
% \end{macro}
% \end{macro}
%
% 其实可以直接写到~\cs{chapter} 的定义里面。
%    \begin{macrocode} 
\renewcommand{\chaptermark}[1]{\markboth{\chaptername~~#1}{}}
\renewcommand{\sectionmark}[1]{\markright{\xd@title}}
%</cls>
%    \end{macrocode}
%
% \changes{v0.9.1}{2009/06/15}{增加首行按照两个中文字符缩进。}
% \subsubsection{段落}
% \label{sec:paragraph}
%
% 用于中文段落缩进和正文版式
%    \begin{macrocode}
%<*cls>
\newlength\CJK@twochars
\def\CJK@spaceChar{\Unicode{48}{7}}
\def\CJKindent{%
  \settowidth\CJK@twochars{\CJK@spaceChar\CJK@spaceChar}%
  \parindent\CJK@twochars}
%    \end{macrocode}
%
% 段落之间的竖直距离
%    \begin{macrocode}
\setlength{\parskip}{0pt \@plus2pt \@minus0pt}
%    \end{macrocode}
%
% 调整默认列表环境间的距离,以符合中文习惯。
% \begin{macro}{xd@item@space}
%    \begin{macrocode}
\def\xd@item@space{%
  \let\itemize\compactitem
  \let\enditemize\endcompactitem
  \let\enumerate\compactenum
  \let\endenumerate\endcompactenum
  \let\description\compactdesc
  \let\enddescription\endcompactdesc}
%</cls>
%    \end{macrocode}
% \end{macro}
%
% \subsubsection{中文标题定义}
% \label{sec:theor}
%
% \changes{v0.2}{2009/06/06}{加入中文标题的定义。}
%
%    \begin{macrocode}
%<*cfg>
\renewcommand\contentsname{目\hspace{1em}录}
\renewcommand\listfigurename{插图索引}
\renewcommand\listtablename{表格索引}
\newcommand\listequationname{公式索引}
\newcommand\equationname{公式}
\renewcommand\bibname{参考文献}
\renewcommand\indexname{索引}
\renewcommand\figurename{图}
\renewcommand\tablename{表}
\newcommand\CJKprepartname{第}
\newcommand\CJKpartname{部分}
\newcommand\CJKthepart{\CJKnumber{\@arabic\c@part}}
\def\xd@CJKnumber#1{\ifcase#1{零}\or%
                    {一}\or{二}\or{三}\or{四}\or{五}\or%
                    {六}\or{七}\or{八}\or{九}\or{十}\or%
                    {十一}\or{十二}\or{十三}\or{十四}\or{十五}\or%
                    {十六}\or{十七}\or{十八}\or{十九}\or{二十}\fi}
\newcommand\CJKprechaptername{第}
\newcommand\CJKchaptername{章}
\ifxd@bachelor
  \newcommand\CJKthechapter{\xd@CJKnumber{\@arabic\c@chapter}}
  \newcommand{\CJKthechaptername}[1]{%
              \CJKprechaptername~\xd@CJKnumber{\@arabic#1}~\CJKchaptername~~}
\fi
\renewcommand\chaptername{\CJKprechaptername~\CJKthechapter~\CJKchaptername}
\newcommand{\cabstractname}{摘\hspace{1em}要}
\newcommand{\eabstractname}{ABSTRACT}
\newcommand{\xd@ackname}{致\hspace{1em}谢}
\newcommand{\xd@ckeywords@title}{关键词:}
%</cfg>
%    \end{macrocode}

% \subsubsection{章节标题}
% \label{sec:titleandtoc}
%    \begin{macrocode}
%<*cls>
\titleformat{\chapter}[block]%
            {\sanhao\hei}{\chaptername}%
            {1ex}{\sanhao\hei\filcenter}%
            [\ifxd@mkabstract\thispagestyle{xd@abstract}\else\thispagestyle{xd@headings}\fi]
\titlespacing*{\chapter}{0pt}{4ex}{3ex}[0pt]
%    \end{macrocode}
% \begin{macro}{\section}
% 一级节标题,例如:2.1  实验装置与实验方法
% 节标题序号与标题名之间空一个汉字符(下同)。
% 采用宋体四号(14pt)字居中书写。
%    \begin{macrocode}
\titleformat{\section}[block]%
            {\sihao[1.429]}{\thesection}%
            {1ex}{\sihao[1.429]\filcenter}
%    \end{macrocode}
% \end{macro}
%
% \begin{macro}{\subsection}
% 二级节标题,例如:2.1.1  实验方法
% 节标题序号与标题名之间空一个汉字符(下同)。
% 采用宋体小四号(12pt)字居左书写。
%    \begin{macrocode}
\titleformat{\subsection}[block]%
            {\xiaosi}{\thesubsection}%
            {1ex}{\xiaosi}
%</cls>
%    \end{macrocode}
% \end{macro}
%
%
%\subsubsection{目录格式}
%\label{sec:tableofcontents}
%
%    \begin{macrocode}
%<*cls>
\let\xd@orgtoc\tableofcontents
\renewcommand\tableofcontents{\xd@orgtoc\markright{\contentsname}}
\titlecontents{chapter}[0pt]{}%
              {\xiaosi\bfseries\song%
                \ifxd@numbered\CJKthechaptername\thecontentslabel\else%
                \thecontentslabel\fi}{}%
              {\titlerule*[.6pc]{.}\contentspage}
%</cls>
%    \end{macrocode}
%
%\subsubsection{数学相关}
%\label{sec:maths}
%
%    \begin{macrocode}
%<*cls>
\renewcommand\theequation{\ifnum \c@chapter>\z@ \thechapter%
                          -\fi\@arabic\c@equation}
%</cls>
%    \end{macrocode}
%
% \subsubsection{浮动对象以及表格}
% \label{sec:float}
%
% 设置浮动对象和文字之间的距离
% \changes{v0.2}{2009/06/06}{增加~\cs{subfloat}}
%    \begin{macrocode}
%<*cls>
\let\old@tabular\@tabular
\def\xd@tabular{\dawu[1.5]\old@tabular}
\DeclareCaptionLabelFormat{xd@cap}{{\dawu[1.5] #1~\rmfamily #2}}
\DeclareCaptionLabelSeparator{xd@sep}{\hspace{1em}}
\DeclareCaptionFont{xd@capfont}{\dawu[1.5]}
\captionsetup{labelformat=xd@cap,labelsep=xd@sep,font=xd@capfont}
\captionsetup[table]{position=top,belowskip={12bp-\intextsep},aboveskip=3bp} 
\captionsetup[figure]{position=bottom,belowskip={12bp-\intextsep},aboveskip=3bp}
\captionsetup[subfloat]{font=xd@capfont,captionskip=6bp,%
                        nearskip=6bp,farskip=0bp,topadjust=0bp}
 % \renewcommand{\thesubfigure}{\thefigure--(\arabic{subfigure})}
 % \renewcommand{\p@subfigure}{:}
%</cls>
%    \end{macrocode}

% \subsubsection{摘要格式}
% \label{sec:abstractformat}
%
% \begin{macro}{\xd@makeabstract}
% 中文摘要部分的标题为"摘要",用黑体三号字。
%    \begin{macrocode}
%<*cls>
\long\@xp\def\@xp\collect@@body\@xp#\@xp1\@xp\end\@xp#\@xp2\@xp{%
  \collect@@body{#1}\end{#2}}
\long\@xp\def\@xp\push@begins\@xp#\@xp1\@xp\begin\@xp#\@xp2\@xp{%
  \push@begins{#1}\begin{#2}}
\long\@xp\def\@xp\addto@envbody\@xp#\@xp1\@xp{%
  \addto@envbody{#1}}
\newcommand{\xd@@cabstract}[1]{\long\gdef\xd@cabstract{#1}}
\newenvironment{cabstract}{\collect@body\xd@@cabstract}{}
\newcommand{\xd@@eabstract}[1]{\long\gdef\xd@eabstract{#1}}
\newenvironment{eabstract}{\collect@body\xd@@eabstract}{}
\newcommand{\xd@@ckeywords}[1]{\long\gdef\xd@ckeywords{#1}}
\newenvironment{ckeywords}{\collect@body\xd@@ckeywords}{}
\newcommand{\xd@@ekeywords}[1]{\long\gdef\xd@ekeywords{#1}}
\newenvironment{ekeywords}{\collect@body\xd@@ekeywords}{}
\newcommand{\xd@makeabstract}{%
  \xd@mkabstracttrue%
% 摘要内容用小四号字书写,两端对齐,汉字用宋体,外文字用~Times New Roman 体,标点
% 符号一律用中文输入状态下的标点符号
  \chapter*{\cabstractname}%
  \markboth{\cabstractname}{}%
  \normalsize\xd@cabstract\vskip12bp%
% 关键词为悬挂缩进
  \setbox0=\hbox{\hei\xd@ckeywords@title\hspace{1em}}%
  \ifxd@bachelor\noindent\hangindent\wd0\hangafter1\fi
  \box0{\hei\xd@ckeywords}%
% 英文摘要部分的标题为“ABSTRACT”,用~Times New Roman 体三号字。
  \chapter*{\bfseries\eabstractname}%
  \markboth{\eabstractname}{}%
  \normalsize\xd@eabstract\vskip12bp%
% 关键词为悬挂缩进
  \setbox0=\hbox{\bfseries Keywords:\hspace{1em}}%
  \ifxd@bachelor\noindent\hangindent\wd0\hangafter1\fi
  \box0{\bfseries\xd@ekeywords}%
  \xd@mkabstractfalse%
}
%</cls>
%    \end{macrocode}
% \end{macro}
%
% \begin{macro}{\makecover}
%    \begin{macrocode}
%<cls>\newcommand\makecover{\xd@makeabstract}
%    \end{macrocode}
% \end{macro}
%
% \subsubsection{致谢}
% \label{sec:ack}
%
%    \begin{macrocode}
%<*cls>
\newenvironment{ack}{%
  \xd@numberedfalse%
  \chapter*{\xd@ackname}%
  \addcontentsline{toc}{chapter}{\bfseries\song\xd@ackname}%
  \markboth{\xd@ackname}{}%
  \xd@numberedtrue}%
  {}
%</cls>
%    \end{macrocode}
%
% \subsubsection{参考文献}
% \label{sec:ref}
%
% \begin{macro}{\onlinecite}
% 正文引用模式。依赖于~\pkg{natbib}~宏包,修改其中的命令。
%    \begin{macrocode}
%<*cls>
\bibpunct{[}{]}{,}{s}{}{,}
\renewcommand\NAT@citesuper[3]{\ifNAT@swa
\unskip\kern\p@\textsuperscript{\NAT@@open #1\NAT@@close}%
   \if*#3*\else\ (#3)\fi\else #1\fi\endgroup}
\DeclareRobustCommand\onlinecite{\@onlinecite}
\def\@onlinecite#1{\begingroup\let\@cite\NAT@citenum\citep{#1}\endgroup}
%    \end{macrocode}
% \end{macro}
%
% 参考文献的正文部分用五号字。
% 行距采用固定值~16 磅,段前空~3 磅,段后空~0 磅。
%
% \begin{environment}{thebibliography}
% 修改默认的~thebibliography 环境,增加一些调整代码。
%    \begin{macrocode}
\renewenvironment{thebibliography}[1]{%    
  \xd@numberedfalse
  \chapter*{\bibname}%
  \addcontentsline{toc}{chapter}{\bfseries\song\bibname}%
  \markboth{\bibname}{\xd@title}%
  \xd@numberedtrue%
  \wuhao[1.5]
  \list{\@biblabel{\@arabic\c@enumiv}}%
  {\renewcommand{\makelabel}[1]{##1\hfill}
    \settowidth\labelwidth{1.1cm}
    \setlength{\labelsep}{0.6em}
    \setlength{\itemindent}{0pt}
    \setlength{\leftmargin}{\labelwidth+\labelsep}
    \addtolength{\itemsep}{-0.7em}
    \usecounter{enumiv}%
    \let\p@enumiv\@empty
    \renewcommand\theenumiv{\@arabic\c@enumiv}}%
  \sloppy
  \clubpenalty4000
  \@clubpenalty \clubpenalty
  \widowpenalty4000%
  \interlinepenalty4000%
  \sfcode`\.\@m}
{\def\@noitemerr
  {\@latex@warning{Empty `thebibliography' environment}}%
  \endlist}
%</cls>
%    \end{macrocode}
% \end{environment}
%
%    \begin{macrocode}
%<cls>\AtEndOfClass{\input{xdthesis.cfg}}%
%    \end{macrocode}
%
% 把 CJK 环境放到合适的位置,以免导致其它宏包的命令位于 CJK 环境中而出现问题(比
% 如 natbib 的``Multiple-defined labels'',同时自动开启 CJK。
%    \begin{macrocode}
%<*cls>
\AtBeginDocument{\CJKindent}
\AtEndOfClass{\sloppy\xd@item@space}
%</cls>
%    \end{macrocode}
%
% \Finale
\endinput}%
%    \end{macrocode}
%
% 把 CJK 环境放到合适的位置,以免导致其它宏包的命令位于 CJK 环境中而出现问题(比
% 如 natbib 的``Multiple-defined labels'',同时自动开启 CJK。
%    \begin{macrocode}
%<*cls>
\AtBeginDocument{\CJKindent}
\AtEndOfClass{\sloppy\xd@item@space}
%</cls>
%    \end{macrocode}
%
% \Finale
\endinput}%
%    \end{macrocode}
%
% 把 CJK 环境放到合适的位置,以免导致其它宏包的命令位于 CJK 环境中而出现问题(比
% 如 natbib 的``Multiple-defined labels'',同时自动开启 CJK。
%    \begin{macrocode}
%<*cls>
\AtBeginDocument{\CJKindent}
\AtEndOfClass{\sloppy\xd@item@space}
%</cls>
%    \end{macrocode}
%
% \Finale
\endinput}%
%    \end{macrocode}
%
% 把 CJK 环境放到合适的位置,以免导致其它宏包的命令位于 CJK 环境中而出现问题(比
% 如 natbib 的``Multiple-defined labels'',同时自动开启 CJK。
%    \begin{macrocode}
%<*cls>
\AtBeginDocument{\CJKindent}
\AtEndOfClass{\sloppy\xd@item@space}
%</cls>
%    \end{macrocode}
%
% \Finale
\endinput
%% ----------------------------------------------------------------------
%% START OF FILE
%% ----------------------------------------------------------------------
%% 
%% Filename: ch01-intro.tex
%% Author: Fred Qi
%% Created: 2011-03-14 18:22:25(+0800)
%% 
%% ----------------------------------------------------------------------
%%% CHANGE LOG
%% ----------------------------------------------------------------------
%% Last-Updated: 2011-04-20 12:17:25(+0800) [by Fred Qi]
%%     Update #: 23
%% ----------------------------------------------------------------------

\chapter{引言}
\label{cha:intro}

本模板的示例论文体例结构仅供学习使用本模板之用,并不是撰写学位论文的所必需遵守的
模式。具体写作时请根据自己所做的工作内容组织文章结构。

\section{论文工作背景}
\label{sec:motivation}

在这里介绍论文工作的目标,即做什么,为什么做。

\section{相关工作}
\label{sec:related-works}

在这里介绍与论文内容相关的他人工作。通常需要引用一些文献。下面内容可以做为引用文
献的例子\footnote{文献格式的工作尚未开始,拟基于\texttt{biblatex}实现文献引用,工
  作量较大。}。

文献\onlinecite{lastname11:_examp_artic}给出了期刊文章的例
子,文献\onlinecite{ln111:_examp_confer_artic_title}则给出了会议文章的例子,另外
一个是学术专著\cite{book11:_examp_book_title}的例子。对期刊与会议文章,为避免期刊
名或会议名过长,常采用其缩写形式。IEEE给出了其期刊的标准缩写格式,如文
献\onlinecite{naseem_linear_2010}所给出的示例。这段内容请给
合~\texttt{refs.bib}~进行阅读。

%%% Local Variables:
%%% TeX-master: "main.tex"
%%% End:
%% ----------------------------------------------------------------------
%%% END OF FILE 
%% ----------------------------------------------------------------------